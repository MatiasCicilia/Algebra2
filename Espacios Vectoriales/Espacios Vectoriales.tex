\documentclass[11pt]{article}
\usepackage[utf8]{inputenc}
\usepackage{amsmath}
\usepackage{color}
% -------------  Header   ---------------
\usepackage{fancyhdr}

\pagestyle{fancy}

\fancyhf{}
\fancyheadoffset{2pt}
\fancyhead[L]{Algebra II}
\fancyhead[R]{Universidad Austral}
% -------------  Margenes ---------------
\addtolength{\oddsidemargin}{-.875in}
	\addtolength{\evensidemargin}{-.875in}
	\addtolength{\textwidth}{1.75in}

	\addtolength{\topmargin}{-.875in}
	\addtolength{\textheight}{1.75in}
% ------------- /Margenes ---------------
% ------------- Espaciado entre lineas ---------------
\renewcommand{\baselinestretch}{1.5} 
% !TeX spellcheck = es_ES
\title{Espacios Vectoriales}
\author{Álgebra II}
\date{\vspace{-5ex}}

\begin{document}
\maketitle{}
% ------------- Comienza el documento ---------------
\section{Definición}
Sea V un conjunto cuyos elementos se llamarán vectores en
el cual se definen dos operaciones:
\begin{itemize}
\item Suma de vectores
\item Producto de un vector por un escalar k $\in {\rm I\!R}$\
\end{itemize}
Estas operaciones cumplen las siguientes propiedades:\\
\underline {Propiedades de la Suma:}
\begin{enumerate}
\item {\bfseries Cerrada:} Si $\vec{u}, \vec{v} \in V \Rightarrow \vec{u}+\vec{v} \in V$
\item {\bfseries Conmutativa:} $\vec{u} + \vec{v} = \vec{v} + \vec{u}$ para todo $\vec{u}, \vec{v} \in V $ 
\item {\bfseries Asociativa:} $(\vec{u}+\vec{v})+\vec{w} = \vec{u}+(\vec{v}+\vec{w})$
\item {\bfseries Elemento Neutro:} $\exists$ $\vec{0} \in V$ / $\vec{0} + \vec{x}$ = $\vec{x}$ $\forall$ $\vec{x}  \in V$
\item {\bfseries Vector Inverso:} $\forall$ $\vec{x} \in$ V, existe un vector inverso -$\vec{x}$ / $\vec{x}+(-\vec{x})=\vec{0}$
\end{enumerate}
\underline{Propiedades del producto:}
\begin{enumerate}
\item {\bfseries Cerrada:} Si $\vec{x} \in V \Rightarrow k\cdot\vec{x} \in V$
\item {\bfseries Neutro:} $1\cdot\vec{x} = \vec{x}$ $\forall$ $\vec{x} \in V $ 
\item {\bfseries Asociativa:} $k_{1}\cdot(k_{2} \cdot \vec{x} ) = (k_{1} \cdot k_{2})\cdot\vec{x}$
\item {\bfseries Distributiva:} $k  \cdot (\vec{x} + \vec{y}) = k \cdot \vec{x} + k \cdot \vec{y} $
\item {\bfseries Distributiva:} $(k_{1} + k_{2})\cdot \vec{x} = k_{1}\cdot\vec{x} + k_{2}\cdot\vec{x}$
\end{enumerate}
% ------------- Subespacios ---------------
\section{Subespacios} 
{\bfseries Definición:} Un subconjunto S no vacío de V e.v.es un subespacio de V si la suma y el producto definidas en  V estructuran también a S como un espacio vectorial. \\
\underline {Propiedades necesarias para que S $\subseteq$ V sea subespacio}
\begin{enumerate}
\item Si $\vec{x} \in S$ e $\vec{y} \in S \Rightarrow \vec{x} + \vec{y} \in S$
\item Si $\vec{x} \in S \Rightarrow k \cdot \vec{x} \in S$
\item $\vec{0} \in S$
\end{enumerate}
\subsection{Definiciones}
\begin{enumerate}
\item {\bfseries Combinación Lineal:} $\vec{x}$ es una combinación lineal de los vectores $\vec{x}_{1}, \vec{x}_2,... \vec{x}_k $ si existen escalares $k_{1}, k_{2},... k_{k} $ tal que $\vec{x} = k_{1}\vec{x}_{1} + k_{2}\vec{x}_{2} +... k_{k}\vec{x}_{k}$
\item {\bfseries Sistema de Generadores:} Un conjunto de vectores M = \{$\vec{x}_{1}, \vec{x}_2,... \vec{x}_k $\} es un sistema de generadores de V $\Leftrightarrow \forall $ $ \vec{v} \in V, \vec{v} = \alpha_{1}\vec{x}_{1} + \alpha_{2}\vec{x}_{2} +... \alpha_{k}\vec{x}_{k}$
\item {\bfseries Conjunto de vectores linealmente independiente y linealmente dependiente:}\\
Sea M = \{ $\vec{v}_{1}, \vec{v}_2,\hdots, \vec{v}_k $ \} tal que $\vec{v}_{i} \in$ V $\forall $ i, y sea $\alpha_{1}\vec{x}_{1} + \alpha_{2}\vec{x}_{2} + \hdots + \alpha_{k}\vec{x}_{k} = \vec{0}$ (o sea una combinación lineal de ellos igualada a $\vec{0}$)\\
Nos queda entonces un sistema lineal homogéneo que puede tener:\\
a) \underline{Solución Única:} La unica solución es la trivial, por lo que M es un conjunto {\bfseries Linealmente Independiente (L.I.)}\\
b) \underline{Infinitas Soluciones:} M es un conjunto {\bfseries Linealmente Dependiente (L.D.)}
\end{enumerate}
\newpage
% ------------- Base ---------------
\section{Base de un Espacio Vectorial}
{\bfseries Definición:} Dado V e.v. y B = \{$\vec{v}_{1}, \vec{v}_2,... \vec{v}_n $\} tal que $\forall$ $\vec{v}_{i} \in$ V decimos que B es una base de V si y sólo si se cumple:\\
a) B es un conjunto L.I.\\
b) B es un sistema de generadores de V \\
Ejemplos de bases canónicas: 
\begin{itemize}
\item {\bfseries En R$^{n}$:} (1,0,0,...,0), (0,1,0,...,0), ... , (0,0,0,...,1) 
\item {\bfseries En R$^{2x2}$:} $\begin{pmatrix} 1&0\\ 0&0 \end{pmatrix}$ , $\begin{pmatrix} 0&1\\ 0&0 \end{pmatrix}$ , $\begin{pmatrix} 0&0\\ 1&0 \end{pmatrix}$ , $\begin{pmatrix} 0&0\\ 0&1 \end{pmatrix}$ 
\item {\bfseries En P$_{n}$:} \{1, x, x$^2$, ... , x$^n$ \} 
\end{itemize}
% ------------- Esto lo movi de su lugar original. Preguntar ---------------
\subsection{Dimensión de un Espacio Vectorial}
{\bfseries Definición:} Se llama dimensión de un espacio vectorial V a la cantidad de elementos que tiene una base.
% ------------- Coordenadas ---------------
\section{Coordenadas de un vector respecto de una base}
Dado B = \{$\vec{e}_{1}, \vec{e}_2,... \vec{e}_n $\} una base de V (e.v.) y dado $\vec{x} \in V$ $\Rightarrow$\\ 
Se denomina a los escalares $\lambda_{1}, \lambda_{2},... \lambda_{n}$ tal que $\vec{x} = \lambda_{1}\vec{e}_{1} + \lambda_{2}\vec{e}_{2} +... \lambda_{k}\vec{e}_{k}$ como {\bfseries coordenadas de} $\vec{x} $ respecto de la base B y se escribe $\vec{x}]_B$ \\
Por lo tanto: $\vec{x}]_B$ = ($\lambda_{1}, \lambda_{2},\hdots, \lambda_{n}$)
\newpage
% ------------- Proposiciones ---------------
\section{Proposiciones}
% ------------- Coordenadas Unicas ---------------
\subsection{Las coordenadas de un vector respecto de una base 
son únicas}
{\bfseries {Hipótesis:}} V e.v. \{$\vec{e}_{1}, \vec{e}_2,... \vec{e}_n $\} base de V, $\vec{x} \in V$ \\
{\bfseries {Tesis:}} Existe una única n-upla ($\alpha_{1}, \alpha_{2},... \alpha_{n}$) tal que $\vec{x} = \alpha_{1}\vec{e}_{1} + \alpha_{2}\vec{e}_{2} +... \alpha_{n}\vec{e}_{n}$\\
{\bfseries Demostración:} Supongo que existen 2 n-uplas que satisfacen la definición de coordenadas respecto de una base, es decir:\\
$\vec{x} = \alpha_{1}\vec{e}_{1} + \alpha_{2}\vec{e}_{2} +... \alpha_{n}\vec{e}_{n} = \beta_{1}\vec{e}_{1} + \beta_{2}\vec{e}_{2} +... \beta_{n}\vec{e}_{n}$\\
Pasando al segundo miembro y sacando factor común $\vec{e}_i$ nos queda: \\
$(\alpha_{1} - \beta_{1})\cdot\vec{e}_{1} + (\alpha_{2} - \beta_{2})\cdot\vec{e}_{2} +... (\alpha_{n} -\beta_{n})\cdot\vec{e}_{n} = \vec{0}$\\
Como \{$\vec{e}_{1}, \vec{e}_2,... \vec{e}_n $\} es una base de V, es decir se trata de vectores L.I. Esto significa que si tengo una combinación lineal de ellos igualada a $\vec{0}$, sus escalares son = 0. \\
Por lo tanto, $\alpha_{i} = \beta_{i}$ $\forall$ i.
% ------------- n+1 es LD --------------- 
\subsection{Si un espacio vectorial tiene una  base de n 
elementos , entonces cualquier conjunto de n+1 
elementos es un conjunto l.d.}
{\bfseries {Hipótesis:}} V e.v. B = \{$\vec{e}_{1}, \vec{e}_2,... \vec{e}_n $\} base de V, \{$\vec{x}_{1}, \vec{x}_2,\hdots, \vec{x}_n, \vec{x}_{n+1} $\} conjunto de n+1 elementos\\
{\bfseries {Tesis:}} \{$\vec{x}_{1}, \vec{x}_2,\hdots, \vec{x}_n, \vec{x}_{n+1} $\} es L.D.\\
{\bfseries Demostración:} Para demostrar que un conjunto es L.I.  debo probar que cualquier combinación lineal de sus vectores igualada a $\vec{0}$ tiene una única solución (La solución trivial). Para probar que es L.D., una combinación lineal de dichos vectores igualada a $\vec{0}$ tendrá infinitas soluciones. \\ Hagamos entonces: 
\begin{align}
\lambda_1\vec{x}_1+\lambda_2\vec{x}_2+\hdots+\lambda_{n+1}\vec{x}_{n+1} = \vec{0}
\end{align}
Comos los $\vec{x}_i$ son vectores pertenecientes a V, y B es una base de de V, entonces cada uno de ellos podrá escribirse como combinación lineal de los vectores de V:\\
$\vec{x}_1=\alpha_{11}\cdot\vec{e}_1+\alpha_{12}\cdot\vec{e}_2+\hdots+\alpha_{1n}\cdot\vec{e}_n$\\
$\vec{x}_2=\alpha_{21}\cdot\vec{e}_1+\alpha_{22}\cdot\vec{e}_2+\hdots+\alpha_{2n}\cdot\vec{e}_n$\\
$\vdots$\\
$\vec{x}_{n+1}=\alpha_{n+1,1}\cdot\vec{e}_1+\alpha_{n+1,2}\cdot\vec{e}_2+\hdots+\alpha_{n+1,n}\cdot\vec{e}_n$\\
Reemplazando en la ecuación (1) nos queda:\\
$\lambda_1\cdot(\alpha_{11}\cdot\vec{e}_1+\alpha_{12}\cdot\vec{e}_2+\hdots+\alpha_{1n}\cdot\vec{e}_n) + \lambda_2\cdot(\alpha_{21}\cdot\vec{e}_1+\alpha_{22}\cdot\vec{e}_2+\hdots+\alpha_{2n}\cdot\vec{e}_n) + \hdots \lambda_{n+1}\cdot(\alpha_{n+1,1}\cdot\vec{e}_1+\alpha_{n+1,2}\cdot\vec{e}_2+\hdots+\alpha_{n+1,n}\cdot\vec{e}_n)$\\
Sacando factor $\vec{e}_i$ nos queda:
$(\lambda_1\cdot\alpha_{11} + \lambda_2\cdot\alpha_{21} + \hdots + \lambda_{n+1}\cdot\alpha_{n+1,1})\cdot\vec{e}_1 + (\lambda_1\cdot\alpha_{12} + \lambda_2\cdot\alpha_{22} + \hdots + \lambda_{n+1}\cdot\alpha_{n+1,2})\cdot\vec{e}_2 + \hdots + (\lambda_1\cdot\alpha_{1n} + \lambda_2\cdot\alpha_{2n} + \hdots + \lambda_{n+1}\cdot\alpha_{n+1,n})\cdot\vec{e}_n$\\
Como tenemos una combinación lineal igualada a cero de vectores L.I. sus escalares serán todos iguales a 0\\
Entonces nos queda: \\
$\lambda_1\cdot\alpha_{11} + \lambda_2\cdot\alpha_{21} + \hdots + \lambda_{n+1}\cdot\alpha_{n+1,1} = 0$\\
$\lambda_1\cdot\alpha_{12} + \lambda_2\cdot\alpha_{22} + \hdots + \lambda_{n+1}\cdot\alpha_{n+1,2} = 0$\\
$\vdots$\\
$\lambda_1\cdot\alpha_{1n} + \lambda_2\cdot\alpha_{2n} + \hdots + \lambda_{n+1}\cdot\alpha_{n+1,n}=0$\\
Observamos que nos quedan {\bfseries n} ecuaciones.\\
A su vez, los $\alpha_{ij}$ son datos, pues son las coordenadas de los vectores dados (las cuales son únicas)\\
Por otro lado, los $\lambda_i$ son incógnitas, y hay \textbf{n + 1}.\\
Es decir, tenemos un sistema con más incógnitas que ecuaciones y por lo tanto la ecuación (1) tiene {\bfseries infinitas} soluciones. \\
Concluimos entonces que el conjunto \{$\vec{x}_{1}, \vec{x}_2,\hdots, \vec{x}_n, \vec{x}_{n+1} $\} es L.D.
% ------------- Mismo # elementos en distintas bases --------------- 
\subsection{Todas las bases de un mismo espacio vectorial tienen la misma cantidad de elementos}
{\bfseries {Hipótesis:}} V e.v. \{$\vec{v}_{1}, \vec{v}_2,... \vec{v}_n $\} y \{$\vec{e}_{1}, \vec{e}_2,... \vec{e}_p $\} bases de V\\
{\bfseries {Tesis:}} Ambas bases tienen la misma cantidad de elementos. (n = p)\\
{\bfseries Demostración:} Supongo $n>p$. Como tengo una base de p elementos, si tengo un conjunto con al menos un elemento mas, será L.D.\\
Entonces \{$\vec{v}_{1}, \vec{v}_2,... \vec{v}_n $\} sería un conjunto L.D. Absurdo pues al ser una base debe ser un conjunto L.I.\\
Lo mismo ocurre al asumir $p>n$. Por lo tanto la única opción posible es {\bfseries n = p}
\subsection{Si un conjunto de vectores pertenecientes a un e.v. es un conjunto l.d.entonces alguno de ellos es combinación lineal de los demás }
{\bfseries {Hipótesis:}} V e.v. A = \{$\vec{e}_{1}, \vec{e}_2,... \vec{e}_n $\} conjunto L.D. de vectores pertenecientes a V \\
{\bfseries {Tesis:}} $\exists$ j tal que $\vec{e}_j = \alpha_1\cdot\vec{e}_1 + \alpha_2\cdot\vec{e}_2 + \hdots + \alpha_{j-1}\cdot\vec{e}_{j-1} + \alpha_{j+1}\cdot\vec{e}_{j+1} + \hdots + \alpha_n\vec{e}_n$\\
{\bfseries Demostración:} Como A es un conjunto L.D. , dada una combinación lineal de ellos igualada a 0  , el sistema homogéneo que obtengo tendrá infinitas soluciones. \\
Entonces, dada $\alpha_1\vec{e}_1+\alpha_2\vec{e}_2+\hdots+\alpha_n\vec{e}_{n} = \vec{0}$ existirá una n-upla ($\alpha_1,\alpha_2,\hdots,\alpha_n)$ no todos nulos que satisfagan la ecuación.\\
Supongamos $\alpha_1 \neq 0$ entonces podré despejar \\
$\vec{e}_1 = -\frac{\alpha_2}{\alpha_1} \vec{e}_2 - \hdots - \frac{\alpha_n}{\alpha_1}\vec{e}_n$\\
Por lo que ha quedado expresado uno de los vectores como combinación lineal de los demás. 
\subsection{Si un S.G. de un espacio vectorial V es L.D. entonces existe un subconjunto de él de n-1 elementos que es S.G. de V .}
{\bfseries {Hipótesis:}} V e.v. B= \{$\vec{e}_{1}, \vec{e}_2,... \vec{e}_n $\} S.G. de V, B conjunto L.D. \\
{\bfseries {Tesis:}} $\vec{e}_1 , \vec{e}_2 , \hdots , \vec{e}_{j-1} , \vec{e}_{j+1} , \hdots , \vec{e}_n$ S.G. de V\\
{\bfseries Demostración:} Quiero probar que algo es S.G. de V, entonces tomo un $\vec{x}$ $\in$ V. Entonces seguro\\ $\vec{x}$ = $\beta_1\vec{e}_1+\beta_2\vec{e}_2+\cdots+\beta_n\vec{e}_n$ (¿Por qué?). Como B es un conjunto L.D. entonces uno de los vectores será combinación lineal de los demás. Supongamos $\vec{e}_1$ o sea: \\
$\vec{e}_1$ = $-\dfrac{\alpha_2}{\alpha_1}\cdot\vec{e}_2 - \cdots - \dfrac{\alpha_n}{\alpha_1}\cdot\vec{e}_n$ . \\
Reemplazando en $\vec{x}$ nos queda: \\
$\vec{x}$ = $\beta_1 \cdot (-\dfrac{\alpha_2}{\alpha_1}\cdot\vec{e}_2 - \cdots - \dfrac{\alpha_n}{\alpha_1}\cdot\vec{e}_n)+\beta_2\vec{e}_2+\cdots+\beta_n\vec{e}_n $. \\
Sacando factor común queda expresado\\ $\vec{x}$ = $\vec{e}_2 \cdot (- \dfrac{\alpha_2}{\alpha_1} \cdot \beta_1 + \beta_2) + \vec{e}_3 \cdot (- \dfrac{\alpha_3}{\alpha_1} \cdot \beta_1 + \beta_3) + \hdots + \vec{e}_n \cdot (-\dfrac{\alpha_n}{\alpha_1} \cdot \beta_1 + \beta_n)$  \\
De esta manera, $\vec{x}$ queda generado por \{$\vec{e}_2,\hdots,\vec{e}_n\}$, el cual tiene $n-1$ elementos. 
\subsection{Dado V e.v. dim(V) = n. Si tenemos n vectores L.I. pertenecientes a V, son una base de V}
{\bfseries {Hipótesis:}} V e.v, dim(V) = n. \{$\vec{e}_{1}, \vec{e}_2,... \vec{e}_n $\} conjunto L.I. \\
{\bfseries {Tesis:}} \{$\vec{e}_{1}, \vec{e}_2,... \vec{e}_n $\} Base de V\\
{\bfseries Demostración:} Para que un conjunto sea base de V ,debe ser l.i. y SG de V. \\
Que son L.I. lo sabemos por Hip. Sólo nos falta probar que son S.G. de V\\
Para eso debemos probar que cualquier $\vec{x}$ perteneciente a V está generado por dichos vectores. \\
Sea $\vec{x}$ $\in$ V (¿Por qué comienzo así?) y formo el conjunto A = \{$\vec{e}_{1}, \vec{e}_2,... \vec{e}_n, \vec{x} $\} este conjunto es L.D. (¿Por qué?)\\
Entonces existe una C.L. de sus vectores igualada a cero con sus escalares no todos nulos. \\
Sea $\alpha_1\vec{e}_1+\alpha_2\vec{e}_2+\hdots+\alpha_n\vec{e}_{n} + \alpha_{n+1}\vec{x} = \vec{0}$ $\Rightarrow$ puede ser $\alpha_{n+1} = 0$ ó $\alpha_{n+1} \neq 0$\\
Si $\alpha_{n+1} = 0$ nos queda $\alpha_1\vec{e}_1+\alpha_2\vec{e}_2+\hdots+\alpha_n\vec{e}_{n}= \vec{0}$. 
\\Y, como los $\vec{e}_i$ son L.I. tendríamos $\alpha_1 = \alpha_2 = \hdots = \alpha_n = 0$. Lo cual es absurdo (¿Por qué?)\\
Por lo tanto es $\alpha_{n+1} \neq 0$ y podemos despejar.\\
Nos queda $\vec{x}$ = $-\vec{e}_1 \cdot (\dfrac{\alpha_1}{\alpha_{n+1}}) - \vec{e}_2 \cdot (\dfrac{\alpha_{2}}{\alpha_{n+1}}) - \hdots - \vec{e}_n \cdot (\dfrac{\alpha_n}{\alpha_{n+1}})$\\
Por lo que queda demostrado que: \\
\{$\vec{e}_{1}, \vec{e}_2,... \vec{e}_n $\} es una base de V.
\subsection{Dado V e.v. tal que dim(V) = n. Si tenemos un conjunto de n vectores S.G. de V son una base de V}
{\bfseries {Hipótesis:}} V e.v, dim(V) = n. \{$\vec{e}_{1}, \vec{e}_2,... \vec{e}_n $\} S.G. de V \\
{\bfseries {Tesis:}} \{$\vec{e}_{1}, \vec{e}_2,... \vec{e}_n $\} Base de V\\
{\bfseries Demostración:} Debemos demostrar que A = \{$\vec{e}_{1}, \vec{e}_2,... \vec{e}_n $\} es un conjunto L.I. \\
Supongamos que A no es un conjunto l.i. entonces un subconjunto de él continuará siendo un S.G. de V (¿Por qué?). Llamemos $A_1$ a dicho conjunto.\\
$A_1$ no puede ser L.I. porque si lo fuera sería una base de V
de n-1 elementos. Absurdo. (¿Por qué?)\\
Luego $A_1$ debe ser l.d. Entonces un subconjunto de él continuará siendo un S.G. de V . Llamemos $A_2$ a dicho conjunto.\\
Y así podemos continuar hasta llegar a tener $A_n$ =\{$\vec{e}_j$\} un subconjunto de A con un solo elemento que es distinto de 0 (¿Por qué?) por lo
tanto es L.I. o sea que sería una base de V. Absurdo. \\
Por lo tanto A debe ser L.I. y entonces es base de V 
\section{Definiciones}
Dados S y T subespacios de V e.v. podemos definir :
\begin{itemize}
\item {\bfseries {Intersección:}} $S \cap T$ = \{$\vec{x} \in V$ / $\vec{x} \in S$ y $\vec{x} \in T$\}
\item {\bfseries {Unión:}} $S \cup T$ = \{$\vec{x} \in V$ / $\vec{x} \in S$ o $\vec{x} \in T$\}
\item {\bfseries {Suma:}} $S + T$ = \{$\vec{x} \in V$ / $\vec{x} = \vec{a} + \vec{b}$ con $\vec{a} \in S$ y $\vec{b} \in T$\}
\item {\bfseries {Suma directa:}} $S \oplus T$ = $S+T$ con $S \cap T$ = \{$\vec{0}$\}
\end{itemize}
\subsection{Si S y T son subespacios de V e.v. entonces S+T es un
subespacio de V.}
{\bfseries {Hipótesis:}} V e.v. , S y T subespacios de V \\
{\bfseries {Tesis:}} S + T es un subespacio de V\\
{\bfseries {Demostración:}} Es evidente que $S + T \subset V$ (¿Por qué)?\\
Veamos que se cumplen las 3 condiciones para que sea subespacio\\
{\bfseries {a) Sea $\vec{x} \in S+T$ e $\vec{y} \in S+T$}} entonces\\
$\vec{x} = \vec{a} + \vec{b}$ con $\vec{a} \in S$, $\vec{b} \in T$\\
$\vec{y} = \vec{c} + \vec{d}$ con $\vec{c} \in S$, $\vec{d} \in T$\\
Entonces $\vec{x} + \vec{y}$ = $\vec{a}+\vec{b}+\vec{c}+\vec{d}$ = $(\vec{a} + \vec{c}) $ + $(\vec{b} + \vec{d})$\\
Pero $\vec{a}+\vec{c} \in S$ y $\vec{b}+\vec{d} \in T$ (¿Por qué?) $\Rightarrow$ $\vec{x} + \vec{y} $ $\in$ $ S + T $ \\
{\bfseries {b) Sea $\vec{x} \in S+T$}}, entonces $\vec{x} = \vec{a} + \vec{b} $ con $ \vec{a} \in S$ y $\vec{b} \in T$\\
$\alpha\vec{x} = \alpha\cdot (\vec{a} + \vec{b}) = \alpha\vec{a} + \alpha\vec{b}$ (¿Por qué?)\\
Pero $\alpha\vec{a} \in S$ y $\alpha\vec{b} \in T$ $\Rightarrow$ $\alpha\vec{x} \in S+T$\\
{\bfseries {c)}} $\vec{0}_v \in S, \vec{0}_v \in T$ y como $\vec{0}_v + \vec{0}_v = \vec{0}_v$ $\Rightarrow$ $\vec{0}_v \in S+T$\\
Entonces queda probado que S + T es subespacio de V
\subsection{Teorema de la dimension de suma de subespacios}
Si S y T son subespacios de V e.v. ,entonces\\
dim($S+T$) = dim($S$) + dim($T$) - dim($S \cap T$)\\
{\bfseries {Hipótesis:}} V e.v. S y T subespacios de V  \\
{\bfseries {Tesis:}} dim($S+T$) = dim($S$) + dim($T$) - dim($S \cap T$)\\
{\bfseries {Demostración:}} Sea $\{\vec{v}_1, \vec{v}_2, \hdots, \vec{v}_r\}$ una base de $S \cap T$\\
Completamos a una base de S y a una base de T y nos queda: \\
$\{\vec{v}_1, \vec{v}_2, \hdots, \vec{v}_r, \vec{x}_1, \vec{x}_2, \hdots, \vec{x}_p\}$ base de S \\
$\{\vec{v}_1, \vec{v}_2, \hdots, \vec{v}_r, \vec{y}_1, \vec{y}_2, \hdots, \vec{y}_d\}$ base de T \\
Con estos vectores formemos un conjunto que pueda ser base de S+T\\
Por de pronto no deberá tener vectores repetidos (¿Por qué?)\\
Entonces tomamos A = $\{\vec{v}_1, \vec{v}_2, \hdots, \vec{v}_r, \vec{x}_1, \vec{x}_2, \hdots, \vec{x}_p, \vec{y}_1, \vec{y}_2, \hdots, \vec{y}_d\}$\\
Vamos a demostrar que A es un conjunto l.i. y S.G. de S+T\\
Primero veamos que son L.I.\\
Hacemos una C.L. igualada a $\vec{0}$, o sea: \\
$\alpha_1\vec{v}_1 + \alpha_2\vec{v}_2 + \hdots + \alpha_r\vec{v}_r + \beta_1\vec{x}_1 + \beta_2\vec{x}_2 + \hdots + \beta_p\vec{x}_p + \gamma_1\vec{y}_1 + \gamma_2\vec{y}_2 + \hdots + \gamma_d\vec{y}_d = \vec{0}$\\
Lo que es lo mismo que: 
\begin{align}
\alpha_1\vec{v}_1 + \alpha_2\vec{v}_2 + \hdots + \alpha_r\vec{v}_r + \beta_1\vec{x}_1 + \beta_2\vec{x}_2 + \hdots + \beta_p\vec{x}_p  = - \gamma_1\vec{y}_1 - \gamma_2\vec{y}_2 - \hdots - \gamma_d\vec{y}_d
\end{align}
Llamaremos $\vec{h}$ a estos vectores que son iguales. Pero $\vec{h} \in S$ y $\vec{h} \in T$ entonces $\vec{h} \in S \cap T$, o sea estará generado por la base de $S \cap T$. \\
Entonces nos queda: \\
$- \gamma_1\vec{y}_1 - \gamma_2\vec{y}_2 - \hdots - \gamma_d\vec{y}_d = \theta \vec{v}_1 + \theta\vec{v}_2 + \hdots + \theta\vec{v}_r $\\
Si pasamos todo para un mismo miembro:\\
$ \theta \vec{v}_1 + \theta\vec{v}_2 + \hdots + \theta\vec{v}_r + \gamma_1\vec{y}_1 + \gamma_2\vec{y}_2 + \hdots + \gamma_d\vec{y}_d = \vec{0}$\\
Como se trata de una C.L. de vectores L.I. (¿Por qué?) igualada a $\vec{0}$ sus escalares serán todos = 0.\\
O sea será $\vec{h} = \vec{0}$ y $\gamma_1 = \gamma_2 = \hdots = \gamma_d = 0$\\Reemplazando en (1) nos queda:  \\
$\alpha_1\vec{v}_1 + \alpha_2\vec{v}_2 + \hdots, \alpha_r\vec{v}_r + \beta_1\vec{x}_1 + \beta_2\vec{x}_2 + \hdots \beta_p\vec{x}_p  = \vec{0}$\\
Como se trata de vectores L.I. (¿Por qué?) sus escalares son = 0. De esta manera todos los $\alpha_i$ y $\beta_i$ son cero, por lo que queda demostrado que A es un conjunto L.I.\\
Ahora veamos que A es un S.G. de S+T. \\
Sea $\vec{x} \in S+T$ (¿Por qué comienzo así?) $\Rightarrow$ $\vec{x} = \vec{a} + \vec{b}$ con $\vec{a} \in S$ y $\vec{b} \in T$. \\
Pero como tenemos las bases de S y T podemos escribir: \\
$\vec{x} = \alpha_1 \vec{v}_1 + \hdots + \alpha_r \vec{v}_r + \beta_1\vec{x}_1 + \hdots + \beta_p \vec{x}_p + \lambda_1\vec{v}_1 + \hdots + \lambda_r\vec{v}_r + \gamma\vec{x}_1 + \hdots + \gamma_d \vec{y}_d $\\
$\vec{x} = (\alpha_1 + \lambda_1) \vec{v}_1 + \hdots + (\alpha_r + \lambda_r) \vec{v}_r + \beta_1\vec{x}_1 + \hdots + \beta_p \vec{x}_p + \lambda_1\vec{x}_1 + \hdots + \lambda_r\vec{x}_r$\\
Con lo que queda comprobado que A es un S.G.
\end{document}
