\documentclass[11pt]{article}
\usepackage[utf8]{inputenc}
\usepackage{amsmath}
\usepackage{color}

% -------------  Header   ---------------
\usepackage{fancyhdr}

\pagestyle{fancy}

\fancyhf{}
\fancyheadoffset{2pt}
\fancyhead[L]{Algebra II}
\fancyhead[R]{Universidad Austral}
% -------------  Margenes ---------------
\addtolength{\oddsidemargin}{-.875in}
	\addtolength{\evensidemargin}{-.875in}
	\addtolength{\textwidth}{1.75in}

	\addtolength{\topmargin}{-.875in}
	\addtolength{\textheight}{1.75in}
% ------------- /Margenes ---------------
% ------------- Espaciado entre lineas ---------------
\renewcommand{\baselinestretch}{1.5} 
% !TeX spellcheck = es_ES
\title{Demostraciones para el Parcial}
\author{Álgebra II}
\date{\vspace{-5ex}}

\begin{document}
\maketitle{}
% ------------- Comienza el documento ---------------
\section{Espacios Vectoriales}

\begin{itemize}
\item Las coordenadas de un vector respecto de una base son únicas 
\item Si un espacio vectorial tiene una base de $n$ elementos cualquier conjunto con $n+1$ es L.D.
\item Todas las bases de un mismo espacio vectorial tienen la misma cantidad de elementos
\item  Si un conjunto de vectores es L.D. entonces alguno es Combinación lineal de los otros 
\item  Si $\{ \vec{e}_1,\vec{e}_2,\hdots, \vec{e}_n\}$ es un S.G. de un espacio vectorial V y es un conjunto L.D. entonces existe un
subconjunto de él de $n-1$ elementos que es S.G. de V
\item  Dado V e.v. tal que $dim(V)$ = $n$, si tenemos $n$ vectores L.I. $\in$ V $\Rightarrow$ son una base de V 
\item  Dado V e.v $dim(V)$ = $n$, si tenemos un conjunto de n vectores S.G. de V $\Rightarrow$ son una base de V 
\item  Si S y T son subespacios de V e.v. $\Rightarrow$ $S+T$ es un subespacio de V 
\end{itemize}

\section{Transformaciones Lineales}
\begin{itemize}
\item Dado V y W e.v.s y $f$: $V \rightarrow W$ / $f$ es T.L. $\Rightarrow$  $f(\vec{0}_v)$ = $\vec{0}_w$ 
\item Dado V y W e.v.s y $f$: $V \rightarrow W$ / $f$ es T.L. $\Rightarrow$  $Im(f)$ es un subespacio de W
\item Dado V y W e.v.s y $f$: $V \rightarrow W$ / $f$ es T.L. $\Rightarrow$ $f$ es monomorfismo $\leftrightarrow$ $Nu(f)$ = $\{\vec{0}\}$
\item Si $f$: $V \rightarrow W$ es T.L. y $\{ \vec{x}_1,\vec{x}_2,\hdots, \vec{x}_n\}$ base de V $\Rightarrow$  $\{ f(\vec{x}_1),f(\vec{x}_2),\hdots, f(\vec{x}_n)\}$ es S.G. de $Im(f)$
\item Si $f$: $V \rightarrow W$ es T.L. y $\{ \vec{x}_1,\vec{x}_2,\hdots, \vec{x}_n\}$ base de V y $f$ es monomorfismo $\Rightarrow$  $\{ f(\vec{x}_1),f(\vec{x}_2),\hdots, f(\vec{x}_n)\}$ es una base de $Im(f)$
\end{itemize}

\section{Producto Interno}
\begin{itemize}
\item Todo conjunto ortogonal de vectores es un conjunto linealmente independiente.
\item  En todo espacio euclídeo V existe una base ortonormal. (proceso de Gram-Schmidt)
\end{itemize}

\section{Matriz Diagonalizable y Autovectores}
\begin{itemize}
\item A es diagonalizable $\leftrightarrow$ existe una base de autovectores 
\item  Si $\{ \vec{v}_1,\vec{v}_2,\hdots, \vec{v}_n\}$ son autovectores de A $\Rightarrow$ cualquier combinación lineal también lo es 
\end{itemize}
\end{document}