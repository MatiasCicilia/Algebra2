%----------------------------------------------------------%---------------- Sobre la redacción ----------------------
% Con el propósito de aprender a utilizar LaTeX, me propuse
% pasar la teoría de Algebra 2 de la Universidad Austral. 
% Si bien la prolijidad del código no es ideal,
% va mejorando con cada documento.
% Gracias a la profesora Mariana Krause por su ayuda,
% Y a la profesora Cristina Storni por la redacción original.
% Cualquier duda o consulta contactar:
% matias.cicilia@ing.austral.edu.ar
%----------------------------------------------------------
%----------------------------------------------------------
\documentclass[11pt]{article}
\usepackage[utf8]{inputenc}
\usepackage{amsmath}
\usepackage{color}

% -------------  Header   ---------------
\usepackage{fancyhdr}

\pagestyle{fancy}

\fancyhf{}
\fancyheadoffset{2pt}
\fancyhead[L]{Algebra II}
\fancyhead[R]{Universidad Austral}
% -------------  Margenes ---------------
\addtolength{\oddsidemargin}{-.875in}
	\addtolength{\evensidemargin}{-.875in}
	\addtolength{\textwidth}{1.75in}

	\addtolength{\topmargin}{-.875in}
	\addtolength{\textheight}{1.75in}
% ------------- /Margenes ---------------
% ------------- Espaciado entre lineas ---------------
\renewcommand{\baselinestretch}{1.5} 
% !TeX spellcheck = es_ES
\title{Transformaciones Lineales}
\author{Álgebra II}
\date{\vspace{-5ex}}

\begin{document}
\maketitle{}
% ------------- Comienza el documento ---------------
{\Large \bfseries{Definiciones:}}
%------------------------------------------------
%---------------- Sección 1 ---------------------
%------------------------------------------------

\vspace{2mm} \noindent
{\large \bfseries{1) Dado V y W dos espacios vectoriales, decimos que una función $f: V \rightarrow W$ es una transformación lineal u homomorfismo si:}}

\vspace{2mm} \noindent
{\bfseries{a)}} $f(\vec{v}_1+\vec{v}_2)$ = $f(\vec{v}_1)$ + $f(\vec{v}_2)$\\
{\bfseries{b)}} $f(\lambda \cdot \vec{v})$ = $\lambda \cdot f(\vec{v})$

%------------------------------------------------
%---------------- Sección 2 ---------------------
%------------------------------------------------

\vspace{2mm} \noindent
{\large \bfseries{2) Dada $f: V \rightarrow W$ una T.L. definimos:}}

\vspace{2mm} \noindent
{\bfseries{a)}} $N_u(f)$ = \{$\vec{x} \in V$ / $f (\vec{x}) = \vec{0}_w\}$ \\
{\bfseries{b)}} $Im(f)$ = \{$\vec{y} \in W$ / $\exists$ $\vec{x} \in V$ / $f(\vec{x}) = \vec{y}\}$\\
{\bfseries{c)}} Dado S subespacio de V,\\
$f(S)$ = \{$\vec{y} \in W$ / $\exists$ $\vec{x} \in S$ / $f(\vec{x}) = \vec{y}\}$\\
{\bfseries{d)}} Dado T subespacio de W,\\
$f^{-1}(T)$ = $\{\vec{x} \in V$ / $f(\vec{x}) \in W\}$
%------------------------------------------------
%---------------- Sección 3 ---------------------
%------------------------------------------------

\vspace{2mm} \noindent
{\Large \bfseries{3) Dado V y W e.v.s. y $f: V \rightarrow W$ / $f$ es T.L. entonces $f(\vec{0}_v) = \vec{0}_w$}}

\vspace{2mm} \noindent
{\bfseries Hipótesis:} V y W espacios vectoriales, $f: V \rightarrow W$ / $f$ es T.L. \\
{\bfseries Tesis:} $f(\vec{0}_v)$ = $f(\vec{0}_w)$ \\
{\bfseries Demostración:} \\
$\vec{0}_v$ = $0 \cdot \vec{0}_v$ $\Rightarrow$\\
$f(\vec{0}_v)$ = $f(0\cdot \vec{0}_v)$ = $0 \cdot f(\vec{0}_v)$ (¿Por qué?) \\
Pero $f(\vec{0}_v)$ $\in$ W $\Rightarrow$ $0 \cdot f(\vec{0}_v)$ = $\vec{0}_w$

%------------------------------------------------
%---------------- Sección 4 ---------------------
%------------------------------------------------

\vspace{2mm} \noindent
{\Large \bfseries{4) Dado V, W e.v. y $f: V \rightarrow W$ entonces $Im(f)$ es subespacio de W}}

\vspace{2mm} \noindent
{\bfseries Hipótesis:} V y W espacios vectoriales, $f: V \rightarrow W$ / $f$ es T.L. \\
{\bfseries Tesis:} $Im(f)$ es un subespacio de W \\
{\bfseries Demostración:} Debemos probar las 3 condiciones para que sea
subespacio .\\
{\bfseries{a)}} Sean $\vec{x}$ e $\vec{y}$ $\in$ $Im(f)$, \\
Entonces existen $\vec{a}$ y $\vec{b}$ $\in$ V / $f(\vec{a}) = \vec{x}$ y $f(\vec{b})$ = $\vec{y}$ \\
Luego $f(\vec{a} + \vec{b})$ = $f(\vec{a}) + f(\vec{b})$ (¿Por qué?) = $\vec{x} + \vec{y}$\\
Como $\vec{a} + \vec{b}$ $\in$ V (¿Por qué?) entonces $\vec{x} + \vec{y}$ $\in $ $Im(f)$\\
{\bfseries{b)}} Sea $\vec{x} \in Im(f)$, entonces $\exists$ $\vec{a} \in V$ / $f(\vec{a}) = \vec{x}$ \\
Ahora $\alpha \cdot \vec{x}$ = $\alpha \cdot f(\vec{a})$ = $f(\alpha \cdot \vec{a})$ (¿Por qué?)\\
Y como $\alpha \cdot \vec{a}$ $\in$ V (¿Por qué?) $\Rightarrow$ $\alpha \cdot \vec{x} \in Im(f)$ \\
{\bfseries{c)}} Como $f(\vec{0}_v)$ = $\vec{0}_w$ $\Rightarrow$ $\vec{0}_w \in$ W
%------------------------------------------------
%---------------- Sección 5 ---------------------
%------------------------------------------------

\vspace{2mm} \noindent
{\Large \bfseries{5) Si $f: V \rightarrow W$ es una T.L. entonces transforma una base de V en un S.G. de $Im(f)$}}

\vspace{2mm} \noindent
{\bfseries Hipótesis:} $f: V \rightarrow W$ / $f$ es T.L. , $\{\vec{e}_1, \vec{e}_2, \hdots  \vec{e}_n\}$ base de V \\
{\bfseries Tesis:} $\{f(\vec{e}_1), f(\vec{e}_2), \hdots  f(\vec{e}_n)\}$ es S.G. de $Im(f)$ \\
{\bfseries Demostración:} Sea $\vec{x} \in Im(f)$ $\Rightarrow$ $\exists$ $\vec{a} \in V$ / $f(\vec{a}) = \vec{x}$ \\
Como $\vec{a} \in V$ está generado por los elementos de una base de V, se lo puede escribir como: \\
$\vec{a} = \alpha_1\vec{e}_1 + \alpha_2\vec{e}_2 + \hdots + \alpha_n\vec{e}_n$
\\Entonces también: \\
$f(\vec{a})$ = $f(\alpha_1\vec{e}_1 + \alpha_2\vec{e}_2 + \hdots + \alpha_n\vec{e}_n)$\\
Como $f$ es T.L. entonces aplico las dos propiedades y nos queda: \\
$\vec{x}$ = $f(\vec{a})$ = $\alpha_1f(\vec{e}_1) + \alpha_2f(\vec{e}_2) + \hdots + \alpha_nf(\vec{e}_n)$ \\
Entonces $\vec{x}$ está generado por $\{f(\vec{e}_1), f(\vec{e}_2), \hdots  f(\vec{e}_n)\}$
%------------------------------------------------
%---------------- Sección 6 ---------------------
%------------------------------------------------

\vspace{2mm} \noindent
{\Large \bfseries{6) Clasificación}}

\vspace{2mm} \noindent
Si $f:V \rightarrow W$ es una T.L. , entonces
\begin{itemize}
\item $f$ es un monomorfismo $\Leftrightarrow$ $f$ es inyectiva
\item $f$ es un epimorfismo $\Leftrightarrow$ $f$ es suryectiva
\item $f$ es un isomorfismo $\Leftrightarrow$ $f$ es biyectiva
\end{itemize}
%------------------------------------------------
%---------------- Sección 7 ---------------------
%------------------------------------------------

\vspace{2mm} \noindent
{\Large \bfseries{7) $f$ es monomorfismo $\Leftrightarrow$ $N_u(f)$ = $\{\vec{0}\}$}}

\vspace{2mm} \noindent
$\Longrightarrow$ \\
{\bfseries Hipótesis:} V y W e.v.s. y $f: V \rightarrow W$ / $f$ es T.L. \\ $f$ es monomorfismo, o sea si $f(\vec{x}_1)$ = $f(\vec{x}_2)$ $\Rightarrow$ $\vec{x}_1$ = $\vec{x}_2$ \\
{\bfseries Tesis:} $N_u(f)$ = $\{\vec{0}_v\}$ \\
{\bfseries Demostración:} $\vec{0}_v \in N_u(f)$ porque $f(\vec{0}_v)$ = $\vec{0}_w$. \\
Tomemos un $\vec{x} \in N_u(f)$, entonces $f(\vec{x})$ = $\vec{0}_w$. \\
Luego, por ser monomorfismo, $\vec{x}$ = $\vec{0}$. O sea $N_u(f)$ = $\{\vec{0}_v\}$ \\
$\Longleftarrow$ \\
{\bfseries Hipótesis:} V y W e.v.s. y $f: V \rightarrow W$ / $f$ es T.L. y $N_u(f)$ = $\{\vec{0}_v\}$  \\
{\bfseries Tesis:}  $f$ es monomorfismo, o sea si $f(\vec{x}_1)$ = $f(\vec{x}_2)$ $\Rightarrow$ $\vec{x}_1$ = $\vec{x}_2$\\
{\bfseries Demostración:} Sean $\vec{x}_1$ y $\vec{x}_2$ $\in$ V / $f(\vec{x}_1)$ = $f(\vec{x}_2)$\\
Entonces 
$f(\vec{x}_1)$ - $f(\vec{x}_2) = \vec{0}_w$  \\
$\Rightarrow$ $f(\vec{x}_1 - \vec{x}_2)$ = $\vec{0}_w$ (¿Por qué?) \\
$\Rightarrow$ $\vec{x}_1 - \vec{x}_2$ $\in$ $N_u(f)$, pero como $N_u(f)$ = $\{\vec{0}_v\}$  resulta: \\
$\vec{x}_1 - \vec{x}_2$ = $\vec{0}_v$ $\Rightarrow$ $\vec{x}_1$ = $\vec{x}_2$
%------------------------------------------------
%---------------- Sección 8 ---------------------
%------------------------------------------------

\vspace{2mm} \noindent
{\Large \bfseries{8) Si $f$ es un monomorfismo, entonces transforma una base del dominio en una base de $Im(f)$}}

\vspace{2mm} \noindent
{\bfseries Hipótesis:} $f: V \rightarrow W$ / $f$ es monomorfismo (O sea es T.L. y $N_u(f)$ = $\{\vec{0}_v\}$) \\
$\{\vec{e}_1, \vec{e}_2, \hdots  \vec{e}_n\}$ base de V \\
{\bfseries Tesis:} A= $\{f(\vec{e}_1), f(\vec{e}_2), \hdots  f(\vec{e}_n)\}$ es base de $Im(f)$ \\
{\bfseries Demostración:} Debemos probar que A es un conjunto L.I. y S.G. de $Im(f)$ \\
Que es S.G. de $Im(f)$ ya lo sabemos (¿Por qué?). Vamos a probar que A es un conjunto L.I. \\
Sea $\beta_1 f(\vec{e}_1) + \beta_2 f(\vec{e}_2) + \hdots + \beta_n f(\vec{e}_n)$ = $\vec{0}_w$ $\Rightarrow$ \\
$f(\beta_1 \vec{e}_1 + \beta_2 \vec{e}_2 + \hdots + \beta_n \vec{e}_n)$ = $\vec{0}_w$ \\
Entonces $\beta_1 \vec{e}_1 + \beta_2 \vec{e}_2 + \hdots + \beta_n \vec{e}_n$ $\in$ $N_u(f)$. Pero $N_u(f)$ = $\{\vec{0}_v\}$ \\
Por lo tanto $\beta_1 \vec{e}_1 + \beta_2 \vec{e}_2 + \hdots + \beta_n \vec{e}_n$ = $\vec{0}_v$ \\
Y como $\{\vec{e}_1, \vec{e}_2, \hdots  \vec{e}_n\}$ es un conjunto L.I. $\Rightarrow$\\
$\beta_1$ = $\beta_2$ = $\hdots$ = $\beta_n$ = 0\\
Luego A es un conjunto L.I. y por lo tanto base de W.
%------------------------------------------------
%---------------- Sección 9 ---------------------
%------------------------------------------------

\vspace{2mm} \noindent
{\Large \bfseries{9) Teorema de la dimensión de las Transformaciones Lineales}}

\vspace{2mm} \noindent
Si $f: V \rightarrow W$ es una T.L. entonces $dim(N_u(f))$ + $dim(Im(f))$ = $dim(V)$\\
{\bfseries Hipótesis:} $f: V \rightarrow W$ es una T.L. \\
{\bfseries Tesis:} $dim(N_u(f))$ + $dim(Im(f))$ = $dim(V)$ \\
{\bfseries Demostración:} Sea B = $\{\vec{e}_1, \vec{e}_2, \hdots, \vec{e}_r\}$ una base de $N_u(f)$ \\
Extendemos B a una base de V, o sea: \\
B' = $\{\vec{e}_1, \vec{e}_2, \hdots, \vec{e}_r, \vec{f}_1, \vec{f}_2, \hdots, \vec{f}_p\}$ base de V \\
Consideremos el conjunto C igual a los transformados de los elementos de B':\\
C = $\{\vec{0}, \vec{0}, \hdots, \vec{0}, f(\vec{f}_1), f(\vec{f}_2), \hdots, f(\vec{f}_p)\}$\\
C es un S.G. de la imagen (¿Por qué?)\\
Tomamos C' = $\{f(\vec{f}_1), f(\vec{f}_2), \hdots, f(\vec{f}_p)\}$. C' sigue siendo un S.G. de $Im(f)$ porque el $\vec{0}$ no genera nada. \\
Probemos ahora que C' es un conjunto L.I.\\
Sea $\alpha_1 f(\vec{f}_1) + \alpha_2 f(\vec{f}_2) + \hdots + \alpha_p f(\vec{f}_p)$ = $\vec{0}$, entonces: \\
$f(\alpha_1 \vec{f}_1 + \alpha_2 \vec{f}_2 + \hdots + \alpha_p \vec{f}_p)$ = $\vec{0}$ \\
O sea $\alpha_1 \vec{f}_1 + \alpha_2 \vec{f}_2 + \hdots + \alpha_p \vec{f}_p$ $\in$ $N_u(f)$, entonces lo puedo escribir como C.L. de la base de $N_u(f)$: \\
$\alpha_1 \vec{f}_1 + \alpha_2 \vec{f}_2 + \hdots + \alpha_p \vec{f}_p$ = $\beta_1 \vec{e}_1 + \beta_2 \vec{e}_2 + \hdots + \beta_r \vec{e}_r$ \\
Si paso restando me queda: \\
$\alpha_1 \vec{f}_1 + \alpha_2 \vec{f}_2 + \hdots + \alpha_p \vec{f}_p - \beta_1 \vec{e}_1 - \beta_2 \vec{e}_2 - \hdots + \beta_r \vec{e}_r $ = $\vec{0}$ \\
Pero esto es una C.L. igualada a $\vec{0}$ de vectores L.I. (¿Por qué?) \\
Por lo tanto $\alpha_i$ = 0 y $\beta_i$ = 0. $\Rightarrow$ C' es L.I. \\
O sea C' es una base de $Im(f)$.\\
$\Rightarrow$ $dim(V)$ = $r+p$, $dim(N_u(f))$ = r, $dim(Im(f))$ = p.\\
Con lo que queda demostrada la tesis.
%------------------------------------------------
%---------------- Sección 10 ---------------------
%------------------------------------------------

\vspace{2mm} \noindent
{\Large \bfseries{10) Teorema fundamental de las transformaciones lineales}}

\vspace{2mm} \noindent
Sean V y W e.v.s., B = $\{\vec{e}_1, \vec{e}_2, \hdots, \vec{e}_n\}$ base de V y $\{\vec{w}_1, \vec{w}_2, \hdots, \vec{w}_p\}$ elementos cualesquiera de W, entonces {\bfseries existe} una {\bfseries única} T.L. $f: V \rightarrow W$ / $f(\vec{v}_i)$ = $\vec{w}_i$ para todo i. \\
{\bfseries a) Existencia:} \\
{\bfseries Hipótesis:} V y W e.v.s., B = $\{\vec{e}_1, \vec{e}_2, \hdots, \vec{e}_n\}$ base de V y $\{\vec{w}_1, \vec{w}_2, \hdots, \vec{w}_p\}$ elementos de W \\
{\bfseries Tesis:} {\bfseries existe} T.L. $f: V \rightarrow W$ / $f(\vec{v}_i)$ = $\vec{w}_i$ para todo i. \\
{\bfseries Demostración:} Vamos a definir una función $f: V \rightarrow W$. O sea para cada $\vec{x} \in V$ debemos hacerle corresponder un único elemento de W. \\
Sea $\vec{x} \in V$, entonces existen $(\alpha_1, \alpha_2, \hdots \alpha_p)$ únicos tal que: \
$\vec{x}$ = $\alpha_1 \vec{e}_1 + \alpha_2 \vec{e}_2 + \hdots + \alpha_p \vec{e}_p$ (¿Por qué?)\\
Defino $f(\vec{x})$ = $\alpha_1 \vec{w}_1 + \alpha_2 \vec{w}_2 + \hdots + \alpha_p \vec{w}_p$ (¿Está bien definida?) \\
Debemos ahora probar que está función así definida es una T.L. \\
{\bfseries i)} $f(\vec{x} + \vec{y})$ = $f(\vec{x})$ + $f(\vec{y})$ para todo $\vec{x}$ e $\vec{y}$ $\in$ V \\
Sea $\vec{x} \in V$ $\Rightarrow$ $\vec{x}$ = $\delta_1\vec{e}_1 + \delta_2 \vec{e}_2 + \hdots + \delta_p \vec{e}_p$ \\
Entonces $f(\vec{x})$ = $\delta_1\vec{w}_1 + \delta_2 \vec{w}_2 + \hdots + \delta_p \vec{w}_p$ (¿Por qué?) \\
Sea $\vec{y} \in V$ $\Rightarrow$ $\vec{y}$ = $\mu_1\vec{e}_1 + \mu_2 \vec{e}_2 + \hdots + \mu_p \vec{e}_p$ \\
Entonces $f(\vec{y})$ = $\mu_1\vec{w}_1 + \mu_2 \vec{w}_2 + \hdots + \mu_p \vec{w}_p$ \\
Por lo tanto $\vec{x}$ + $\vec{y}$ = $\delta_1\vec{e}_1 + \delta_2 \vec{e}_2 + \hdots + \delta_p \vec{e}_p$ + $\mu_1\vec{e}_1 + \mu_2 \vec{e}_2 + \hdots + \mu_p \vec{e}_p$ = \\
= $(\delta_1 + \mu_1)\vec{e}_1 + (\delta_2 + \mu_2) \vec{e}_2 + \hdots + (\delta_p + \mu_p) \vec{e}_p$\\
Escribimos entonces $f(\vec{x} + \vec{y})$ y $f(\vec{x})$ + $f(\vec{y})$\\
$f(\vec{x} + \vec{y})$ = $(\delta_1 + \mu_1)\vec{w}_1 + (\delta_2 + \mu_2) \vec{w}_2 + \hdots + (\delta_p + \mu_p) \vec{w}_p$ \\
$f(\vec{x}) + f(\vec{y})$ = $\delta_1\vec{w}_1 + \delta_2 \vec{w}_2 + \hdots + \delta_p \vec{w}_p$ + $\mu_1\vec{w}_1 + \mu_2 \vec{w}_2 + \hdots + \mu_p \vec{w}_p$\\
$f(\vec{x}) + f(\vec{y})$ = $(\delta_1 + \mu_1)\vec{w}_1 + (\delta_2 + \mu_2) \vec{w}_2 + \hdots + (\delta_p + \mu_p) \vec{w}_p$\\
Como $f(\vec{x} + \vec{y})$ = $f(\vec{x})$ + $f(\vec{y})$ queda demostrado.\\
{\bfseries ii)} Sea $\vec{h} \in V$ $\Rightarrow$ $\vec{h}$ = $\beta_1\vec{e}_1 + \beta_2 \vec{e}_2 + \hdots + \beta_p\vec{e}_p$ \\
$\alpha \vec{h}$ = $\alpha(\beta_1\vec{e}_1 + \beta_2 \vec{e}_2 + \hdots + \beta_p\vec{e}_p)$ = $(\alpha \beta_1)\vec{e}_1 + (\alpha \beta_2)\vec{e}_2 + \hdots + (\alpha \beta_p)\vec{e}_p$ \\
Luego $f(\alpha \vec{h})$ = $(\alpha \beta_1)\vec{w}_1 + (\alpha \beta_2)\vec{w}_2 + \hdots + (\alpha \beta_p)\vec{w}_p$ \\
Pero $f(\vec{h})$ = $\beta_1\vec{w}_1 + \beta_2 \vec{w}_2 + \hdots + \beta_p\vec{w}_p$\\
Y también $\alpha \cdot f(\vec{h})$ = $\alpha \cdot (\beta_1\vec{w}_1 + \beta_2 \vec{w}_2 + \hdots + \beta_p\vec{w}_p)$ $\Rightarrow$ \\
$\alpha \cdot f(\vec{h})$ = $(\alpha \beta_1)\vec{w}_1 + (\alpha \beta_2)\vec{w}_2 + \hdots + (\alpha \beta_p)\vec{w}_p$\\
Como  $f(\alpha \vec{h})$ = $\alpha \cdot f(\vec{h})$ queda demostrado. \\
Por lo tanto hemos demostrado {\bfseries i)} y {\bfseries ii)} y hemos probado que $f$ así definida es una T.L.\\
Falta probar que esta $f$ cumple con la condición que $f(\vec{e}_i)$ = $\vec{w}_i$ para todo i. \\
Pero $\vec{e}_1$ = $1 \cdot \vec{e}_1 + 0 \cdot \vec{e}_2 + \hdots + 0 \cdot \vec{e}_p$ \\
$\Rightarrow$ $f(\vec{e}_1)$ = $\vec{w}_1$ \\
$\vec{e}_2$ = $0 \cdot \vec{e}_1 + 1 \cdot \vec{e}_2 + \hdots + 0 \cdot \vec{e}_p$ \\
$\Rightarrow$ $f(\vec{e}_2)$ = $\vec{w}_2$ \\
De la misma manera se demuestra que $f(\vec{e}_i)$ = $\vec{w}_i$ \\
{\bfseries b) Unicidad:} \\
Supongamos que existen 2 T.L. $f$ y $g$ $: V \rightarrow W$ / $f(\vec{e}_i)$ = $\vec{w}_i$ y $g(\vec{e}_i)$ = $\vec{w}_i$ \\
Queremos demostrar que $f$ = $g$. \\
O sea $f(\vec{x})$ = $g(\vec{x})$ para todo $\vec{x} \in V$ \\
Sea $\vec{x} \in V$ $\Rightarrow$ $\vec{x}$ = $\theta_1 \vec{e}_1 + \theta_2 \vec{e}_2 + \hdots + \theta_p \vec{e}_p$ \\
$\Rightarrow$ $f(\vec{x})$ = $f(\theta_1 \vec{e}_1 + \theta_2 \vec{e}_2 + \hdots + \theta_p \vec{e}_p)$. \\
$\Rightarrow$ $f(\vec{x})$ = $\theta_1 f(\vec{e}_1) + \theta_2 f(\vec{e}_2) + \hdots + \theta_p f(\vec{e}_p)$. (¿Por qué?) \\
$\Rightarrow$ $f(\vec{x})$ = $\theta_1 \vec{w}_1 + \theta_2 \vec{w}_2 + \hdots + \theta_p \vec{w}_p$. \\
De manera similar planteamos $g(\vec{x})$\\
$\Rightarrow$ $g(\vec{x})$ = $g(\theta_1 \vec{e}_1 + \theta_2 \vec{e}_2 + \hdots + \theta_p \vec{e}_p)$. \\
$\Rightarrow$ $g(\vec{x})$ = $\theta_1 g(\vec{e}_1) + \theta_2 g(\vec{e}_2) + \hdots + \theta_p g(\vec{e}_p)$.  \\
$\Rightarrow$ $g(\vec{x})$ = $\theta_1 \vec{w}_1 + \theta_2 \vec{w}_2 + \hdots + \theta_p \vec{w}_p$. \\
Entonces $f$ = $g$, por lo que la transformación lineal debe ser única.
\end{document}