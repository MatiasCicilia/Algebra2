\documentclass[11pt]{article}
\usepackage[utf8]{inputenc}
\usepackage{amsmath}
\usepackage{color}

% -------------  Header   ---------------
\usepackage{fancyhdr}

\pagestyle{fancy}

\fancyhf{}
\fancyheadoffset{2pt}
\fancyhead[L]{Algebra II}
\fancyhead[R]{Universidad Austral}
% -------------  Margenes ---------------
\addtolength{\oddsidemargin}{-.875in}
	\addtolength{\evensidemargin}{-.875in}
	\addtolength{\textwidth}{1.75in}

	\addtolength{\topmargin}{-.875in}
	\addtolength{\textheight}{1.75in}
% ------------- /Margenes ---------------
% ------------- Espaciado entre lineas ---------------
\renewcommand{\baselinestretch}{1.5} 
% !TeX spellcheck = es_ES
\title{Matriz de Transformación Lineal}
\author{Álgebra II}
\date{\vspace{-5ex}}

\begin{document}
\maketitle{}
% ------------- Comienza el documento ---------------
\section{Matriz de una Transformación Lineal dada una
base de dominio y una base del codominio}
Dada $f : V \rightarrow W$ una T.L. \\
B = $\{\vec{v}_1 , \vec{v}_2, \hdots, \vec{v}_n\}$ una base de V \\
B' = $\{\vec{w}_1, \vec{w}_2, \hdots, \vec{w}_p\}$ una base de W\\
$\Rightarrow$ existe una matriz $A \in {\rm I\!R}^{pxn}$  / para todo $\vec{x} \in V$, $f(\vec{x})]_{B'}$ = $A \cdot \vec{x}]_B$

\noindent
Sea $\vec{x} \in V$ $\Rightarrow$ $\vec{x}$ = $\alpha_1 \vec{v}_1 + \alpha_2 \vec{v}_2 + \hdots + \alpha_n \vec{v}_n$ \\
$\Rightarrow$ $f(\vec{x})$ = $f(\alpha_1 \vec{v}_1 + \alpha_2 \vec{v}_2 + \hdots + \alpha_n \vec{v}_n)$ \\
$\Rightarrow$ $f(\vec{x})$ = $\alpha_1 f(\vec{v}_1) + \alpha_2 f(\vec{v}_2) + \hdots + \alpha_n f(\vec{v}_n)$ (¿Por qué?)\\
Pero, \\
$f(\vec{v}_1)$ = $\beta_{11} \vec{w}_1 + \hdots + \beta_{1p} \vec{w}_p$\\
$f(\vec{v}_2)$ = $\beta_{21} \vec{w}_1 + \hdots + \beta_{2p} \vec{w}_p$\\
$\vdots$ \\
$f(\vec{v}_n)$ = $\beta_{n1} \vec{w}_1 + \hdots + \beta_{np} \vec{w}_p$ \\
Reemplazando en $f(\vec{x})$ = $\alpha_1 f(\vec{v}_1) + \alpha_2 f(\vec{v}_2) + \hdots + \alpha_n f(\vec{v}_n)$ nos queda: \\
$f(\vec{x})$ = 
$\alpha_1 (\beta_{11} \vec{w}_1 + \hdots + \beta_{1p} \vec{w}_p) 
+ \alpha_2 (\beta_{21} \vec{w}_1 + \hdots + \beta_{2p} \vec{w}_p) 
+ \hdots 
+ \alpha_n (\beta_{n1} \vec{w}_1 + \hdots + \beta_{np} \vec{w}_p)$ \\
Aplicando propiedad distributiva y sacando factor común nos queda: \\
$f(\vec{x})$ =
$(\alpha_1 \beta_{11} + \alpha_2 \beta_{21} + \hdots + \alpha_n \beta_{n1}) \vec{w}_1$ + $\hdots$ +  $(\alpha_1 \beta_{1p} + \alpha_2 \beta_{2p} + \hdots + \alpha_n \beta_{np}) \vec{w}_p$. \\
$f(\vec{x})]_{B'}$ = $(\alpha_1 \beta_{11} + \alpha_2 \beta_{21} + \hdots + \alpha_n \beta_{n1}$ , $\alpha_1 \beta_{12} + \alpha_2 \beta_{22} + \hdots + \alpha_n \beta_{n2}$ , $\hdots , \alpha_1 \beta_{1p} + \alpha_2 \beta_{2p} + \hdots + \alpha_n \beta_{np})$ \\
Esto se puede escribir matricialmente: \\

\noindent
$f(\vec{x})]_{B'}$ = 
$\begin{pmatrix} 
\beta_{11} & \beta_{21} & \ldots & \beta_{n1} \\
\beta_{12} & \beta_{22} & \ldots & \beta_{n2}\\
\ldots & \ldots & \ddots & \ldots \\
\beta_{1p} & \beta_{2p} & \ldots & \beta_{np}
\end{pmatrix}$
%Multiplicado por un vector
$\begin{pmatrix} 
\alpha_{1}\\
\alpha_{2}\\
\vdots\\
\alpha_{n}
\end{pmatrix}$ = A $\Rightarrow$ $f(\vec{x})]_{B'}$ = $A \cdot \vec{x}]_B$

\noindent \\
Es importante mirar bien como está formada A. ¿Cuáles son sus columnas? 

\noindent
\section{Matriz de la composición de 2 T.L.}
Si $f_1 : U \rightarrow V$ y $f_2 : V \rightarrow W$ son 2 T.L. y si B, B' y B'' son bases de U, V y W respectivamente, entonces: \\
\textbardbl $f_2 \circ f_1$\textbardbl$_{BB''}$ = 
\textbardbl $f_2$\textbardbl$_{B'B''}$ $\cdot$
\textbardbl $f_1$\textbardbl$_{BB'}$ \\
{\bfseries {Hipótesis:}} \\
$f_1 : U \rightarrow V$ y $f_2 : V \rightarrow W$ son 2 T.L. \\
B, B' y B'' son bases de U, V y W respectivamente \\
\textbardbl $f_2$\textbardbl$_{B'B''}$ y
\textbardbl $f_1$\textbardbl$_{BB'}$ matrices de las T.L. respecto de las bases dadas. \\
{\bfseries {Tesis:}} \\
\textbardbl $f_2 \circ f_1$\textbardbl$_{BB''}$ = 
\textbardbl $f_2$\textbardbl$_{B'B''}$ $\cdot$
\textbardbl $f_1$\textbardbl$_{BB'}$ \\
{\bfseries {Demostración:}} \\
Dado $\vec{x} \in U$ queremos encontrar una matriz A / $f_2 \circ f_1 (\vec{x})]_B''$ = $A \cdot \vec{x}]_B'$ \\
Sabemos que \textbardbl $f_1$\textbardbl$_{BB'}$ $\cdot$ $\vec{x}]_B$ = $f_1(\vec{x})]_{B'}$ . \\
Como son las coordenadas de un vector de V en la base B' puedo hacer el siguiente producto: \\
\textbardbl $f_2$\textbardbl$_{B'B''}$ $\cdot$ $f_1(\vec{x})]_{B'}$ 
= $f_2($ $f_1(\vec{x})$ $)]_{B'}$ o sea\\
$f_2 \circ f_1 (\vec{x})]_{B''}$ =
$f_2($ $f_1(\vec{x})$ $)]_{B'}$ =
\textbardbl $f_2$\textbardbl$_{B'B''}$ $\cdot$
\textbardbl $f_1$\textbardbl$_{BB'}$ $\cdot$ $\vec{x}]_{B}$ \\
$\Rightarrow$ \textbardbl $f_2 \circ f_1$\textbardbl$_{BB''}$ = 
\textbardbl $f_2$\textbardbl$_{B'B''}$ $\cdot$
\textbardbl $f_1$\textbardbl$_{BB'}$
%Sección Definiciones
\section{Definiciones}
\subsection{Transformación Lineal Identidad:}
Sea V e.v. definimos $id: V \rightarrow V$ / $id(\vec{x})$ = $\vec{x}$ , $\forall \vec{x} \in V$
\subsection{Definición de transformación inversa}
Dada $f : V \rightarrow W$ / $f$ es un isomorfismo,\\
Diremos que existe una transformación lineal que llamamos $f^{-1}$ / $f^{-1} : W \rightarrow V$ y se cumple que:
$f^{-1} \circ f (\vec{x})$ = $id(\vec{x})$ , $\forall \vec{x} \in V$ \\
$f \circ f^{-1} (\vec{y})$ = $id(\vec{y})$ , $\forall \vec{y} \in W$ \\
\end{document}