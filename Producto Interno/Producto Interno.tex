\documentclass[11pt]{article}
\usepackage[utf8]{inputenc}
\usepackage{amsmath}
\usepackage{color}
% -------------  Header   ---------------
\usepackage{fancyhdr}

\pagestyle{fancy}

\fancyhf{}
\fancyheadoffset{2pt}
\fancyhead[L]{Algebra II}
\fancyhead[R]{Universidad Austral}
% -------------  Margenes ---------------
\addtolength{\oddsidemargin}{-.875in}
	\addtolength{\evensidemargin}{-.875in}
	\addtolength{\textwidth}{1.75in}

	\addtolength{\topmargin}{-.875in}
	\addtolength{\textheight}{1.75in}
% ------------- /Margenes ---------------
% ------------- Espaciado entre lineas ---------------
\renewcommand{\baselinestretch}{1.5} 
% !TeX spellcheck = es_ES
\title{Producto Interno}
\author{Álgebra II}
\date{\vspace{-5ex}}

\begin{document}
\maketitle{}
% ------------- Comienza el documento ---------------
\section{Definiciones}
\subsection{Producto Interno}
\textbf{Definición:} Un producto interno o escalar sobre un espacio vectorial real V  es una función \\
p: VxV $\rightarrow \rm I\!R $ que cumple con las siguientes propiedades:
\begin{enumerate}
\item $(\vec{x}\cdot\vec{y})=(\vec{y}\cdot\vec{x})$ 
\item $((\vec{x}+\vec{y})\cdot\vec{z})=(\vec{x}\cdot\vec{z})+(\vec{y}\cdot\vec{z})$
\item $(\lambda\cdot\vec{x}\cdot\vec{y}) = \lambda\cdot(\vec{x}\cdot\vec{y})$
\item I) $ $ $(\vec{x}\cdot\vec{x}) \geq 0$ \\ II) $(\vec{x}\cdot\vec{x}) = 0 \leftrightarrow \vec{x}=0 $
\end{enumerate}
\subsection{Espacio Euclideo}
Si V es un e.v. en el que se ha definido un producto interno, recibe el nombre de espacio euclídeo. 
\subsection{Definiciones en un espacio euclideo}
\begin{itemize}
\item Norma de un vector: $\parallel\vec{x}\parallel^2$ = $(\vec{x}\cdot\vec{x})$\\
a) Con cualquier producto interno, el único vector de norma cero es el $\vec{0}$ (¿Por qué?)\\
b) $\parallel\vec{x}\parallel$ = $\parallel-\vec{x}\parallel$ (¿Por qué?)\\
c) $\parallel\alpha\cdot\vec{x}\parallel$ = $\mid\alpha\mid\cdot\parallel\vec{x}\parallel$ (¿Por qué?)
\item Distancia entre dos vectores: \\$dist(\vec{x},\vec{y})$ = $\parallel\vec{x}-\vec{y}\parallel$
\item Ángulo entre dos vectores: \\$\cos \phi = \dfrac{\vec{x}\cdot\vec{y}}{\parallel\vec{x}\parallel\cdot\parallel\vec{y}\parallel}$
\item Ortogonalidad entre vectores: \\
Dos vectores $\vec{x}$ e $\vec{y}$ son ortogonales $\leftrightarrow$ $\vec{x}\cdot\vec{y}$ = $0$
\item Conjunto Ortogonal de Vectores: \\
Un conjunto $\{\vec{e}_1, \vec{e}_2, \hdots, \vec{e}_n\}$ es un conjunto ortogonal $\leftrightarrow$ $\vec{e}_i \neq \vec{0}$ $\forall i$ y $(\vec{e}_i\cdot\vec{e}_j)$ = $0$ $\forall i \neq j$
\item Conjunto Ortonormal de Vectores: \\
Un conjunto $\{\vec{e}_1, \vec{e}_2, \hdots, \vec{e}_n\}$ es ortonormal $\leftrightarrow$ es ortogonal y la norma de sus vectores es 1.
\end{itemize}
% ----------------------- Proposiciones ----------------------
\section{Proposiciones}
\subsection{Teorema de Pitágoras}
Cualquiera sean vectores $\vec{x}$ e $\vec{y}$ vectores pertenecientes a $V e.e.$ tal que $\vec{x}\perp\vec{y}$ se cumple: \\
\centerline {$\parallel\vec{x}+\vec{y}\parallel^2$ = $\parallel\vec{x}\parallel^2$ + $\parallel\vec{y}\parallel^2$}\\
\textbf{Hipótesis:} V e.e. $\vec{x}, \vec{y} \in V$ $\vec{x}\perp\vec{y}$\\
\textbf{Tésis:} $\parallel\vec{x}+\vec{y}\parallel^2$ = $\parallel\vec{x}\parallel^2$ + $\parallel\vec{y}\parallel^2$ \\
\textbf {Demostración:}\\
$\parallel\vec{x}+\vec{y}\parallel^2$ = $(\vec{x} + \vec{y})\cdot(\vec{x} + \vec{y}) \rightarrow$ Expreso la norma como producto \\ 
 = $(\vec{x}\cdot\vec{x}) + (\vec{x}\cdot\vec{y}) + (\vec{y}\cdot\vec{x})+ (\vec{y}\cdot\vec{y}) \rightarrow$ Distribuyo por propiedad 2 \\
 = $\parallel\vec{x}\parallel^2$+  0  +  0  +$\parallel\vec{y}\parallel^2$ (¿Por qué?) \\
O sea hemos llegado a la tesis
\subsection{Desigualdad de Schwarz}
Cualquiera sea $\vec{x}$ e $\vec{y} \in $ V e.e. se cumple: \\
$\mid(\vec{x}\cdot\vec{y})\mid$ $\leq$ $\parallel\vec{x}\parallel \cdot \parallel\vec{y}\parallel $ \\
Esta demostración es necesaria para que esté bien definido el ángulo entre dos vectores (¿Por qué?)\\
\textbf{Hipótesis:} V e.e. $\vec{x}$ e $\vec{y} \in V$ \\
\textbf{Tésis:} $\mid(\vec{x}\cdot\vec{y})\mid$ $\leq$ $\parallel\vec{x}\parallel \cdot \parallel\vec{y}\parallel $ \\
\textbf {Demostración:} \\
Para realizar esta demostración, definimos: $\vec{z} = \vec{x} + \alpha\cdot\vec{y}$ \\
Entonces podemos afirmar $((\vec{x} + \alpha\cdot\vec{y})\cdot(\vec{x} + \alpha\cdot\vec{y})) \geq 0$\\
Aplicando propiedades de producto interno (¿Cuáles?) nos queda: \\
$\parallel\vec{x}\parallel^2 +$ 2 $\alpha\cdot(\vec{x}\cdot\vec{y}) + \alpha^2 \cdot \parallel\vec{y}\parallel^2$ $\geq 0$
\newpage
\noindent Esta expresión representa geométricamente una parábola una parábola de variable $\alpha$, la cual \\ es siempre $\geq$ 0. Entonces tiene 1 o ninguna raíz lo que significa que al aplicar la fórmula para hallar raíces en una parábola:\\\\
$(\dfrac{-b\pm\sqrt{b^2-4ac}}{2a})$ \indent $\bigtriangleup$ = $b^2-4ac \leq 0$
\noindent En nuestro caso resulta: \\
$(2\cdot(\vec{x}\cdot\vec{y}))^2 -4 \cdot \parallel\vec{y}\parallel^2 \parallel\vec{x}\parallel^2$  $\leq 0$\\
$(2\cdot(\vec{x}\cdot\vec{y}))^2$ $\leq 4$ $\cdot \parallel\vec{y}\parallel^2 \parallel\vec{x}\parallel^2$ \\
$\mid(\vec{x}\cdot\vec{y})\mid$ $\leq$ $\parallel\vec{x}\parallel \parallel\vec{y}\parallel$ (¿Por qué?)
\subsection{Todo conjunto ortogonal es L.I.}
\textbf{Hipótesis:} $\{\vec{a}_1, \vec{a}_2, \hdots, \vec{a}_n\}$ Conjunto Ortogonal \\
\textbf{Tésis:} $\{\vec{a}_1, \vec{a}_2, \hdots, \vec{a}_n\}$ Conjunto L.I.\\
\textbf{Demostración:} \\
Sea $\alpha_1\vec{a}_1 + \alpha_2\vec{a}_2 + \hdots + \alpha_n\vec{a}_n$ = $\vec{0}$ \\Debemos probar que $\alpha_1=\alpha_2=\cdots=\alpha_n=0$ \\
Sabemos que $(\vec{a}_i\cdot\vec{a}_j) = 0$ $\forall i \neq j$\\
Multiplicando ambos miembros por $\vec{a}_1$ y aplicando propiedad distributiva en el primer miembro nos queda: 
$\alpha_1\cdot(\vec{a}_1\cdot\vec{a}_1) + \alpha_2\cdot(\vec{a}_1\cdot\vec{a}_2) + \cdots + \alpha_n(\vec{a}_1\cdot\vec{a}_n)$ = $(\vec{a}_1\cdot\vec{0})$\\
$\alpha_1\cdot\parallel\vec{a}_1\parallel^2$ = $0$ y como $\parallel\vec{a}_1\parallel^2 \neq 0$ resulta $\alpha_1 = 0$\\
De la misma manera, multiplicando por los demás vectores nos queda $\alpha_2=\alpha_3=\cdots=\alpha_n=0$\\
Que es lo que queríamos demostrar.
\subsection{Proceso de Grand Smith}
En todo espacio euclídeo V existe una base ortonormal.\\
\textbf{Hipótesis:} V e.e. $\{\vec{e}_1, \vec{e}_2, \hdots, \vec{e}_n\}$ base de V \\
\textbf{Tésis:} $\exists \{\vec{w}_1, \vec{w}_2, \hdots, \vec{w}_n\}$ base ortonormal de V\\
\textbf{Demostración:} Lo vamos a demostrar para $n=3$\\
Sea $\vec{u}_1 = \vec{e}_1$ ; $\vec{u}_2 = \vec{e}_2 + \alpha\cdot\vec{u}_1$ , $\vec{u}_2 \in V$ (¿Por qué?)\\
Queremos determinar $\alpha$ para que $(\vec{u}_1\cdot\vec{u}_2) = 0$ $\rightarrow$ \\
$(\vec{u}_1\cdot\vec{u}_2)$ = $ (\vec{u}_1\cdot\vec{e}_2) + \alpha\cdot\parallel\vec{u}_1\parallel^2 $ = $0$\\\\
Despejando $\alpha$ = $- \dfrac{(\vec{u}_1\cdot\vec{e}_2)}{\parallel\vec{u}_1\parallel^2}$ \indent donde seguro $\parallel\vec{u}_1\parallel^2 \neq 0$ (¿Por qué?) \\\\
Podemos reemplazar y obtener ahora $\vec{u}_2$\\\\
$\vec{u}_2$ = $\vec{e}_2 - \vec{u}_1 \cdot \dfrac{(\vec{u}_1\cdot\vec{e}_2)}{\parallel\vec{u}_1\parallel^2}$\\\\
Ahora armamos
\begin{equation}
\vec{u}_3 = \vec{e}_3 + \theta \cdot \vec{u}_1 + \mu \cdot \vec{u}_2\\
\end{equation} 
Queremos obtener $\theta$ y $\mu$ de manera tal que $\vec{u}_3\perp\vec{u}_1$ y $\vec{u}_3\perp\vec{u}_2$
\noindent Multiplicamos (1) por $\vec{u}_1$ y nos queda: \\
$(\vec{u}_3\cdot\vec{u}_1)$ = $(\vec{e}_3\cdot\vec{u}_1) + \theta\cdot(\vec{u}_1\cdot\vec{u}_1) + \mu \cdot (\vec{u}_1\cdot\vec{u}_2)$ \\
$(\vec{u}_3\cdot\vec{u}_1)$ = $(\vec{e}_3\cdot\vec{u}_1) + \theta\cdot\parallel\vec{u}_1\parallel^2 +$ $0$ (¿Por qué?)\\
Igualando a cero para que sean ortogonales: \\
$(\vec{e}_3\cdot\vec{u}_1) + \theta\cdot\parallel\vec{u}_1\parallel^2$ = 0\\\\
$\rightarrow$ $\theta$ = $- \dfrac{(\vec{e}_3\cdot\vec{u}_1)}{\parallel\vec{u}_1\parallel^2}$\\\\
Ahora multiplicamos (1) por $\vec{u}_2$ y nos queda:\\
$(\vec{u}_3\cdot\vec{u}_2)$ = $(\vec{e}_3\cdot\vec{u}_2) + \theta\cdot(\vec{u}_1\cdot\vec{u}_2) + \mu \cdot (\vec{u}_2\cdot\vec{u}_2)$ \\
$(\vec{u}_3\cdot\vec{u}_2)$ = $(\vec{e}_3\cdot\vec{u}_2)$ + 0 + $\theta\cdot\parallel\vec{u}_2\parallel^2$\\
Igualando a cero para que sean ortogonales: \\
$(\vec{e}_3\cdot\vec{u}_2) + \mu\cdot\parallel\vec{u}_2\parallel^2$ = 0\\\\
$\rightarrow$ $\mu$ = $- \dfrac{(\vec{e}_3\cdot\vec{u}_2)}{\parallel\vec{u}_2\parallel^2}$\\\\
Podemos reemplazar y obtener ahora $\vec{u}_3$: \\\\
$\vec{u}_3$ = $\vec{e}_3$ - $\dfrac{(\vec{e}_3\cdot\vec{u}_1)}{\parallel\vec{u}_1\parallel^2}$ $\cdot\vec{u}_1$ - $\dfrac{(\vec{e}_3\cdot\vec{u}_2)}{\parallel\vec{u}_2\parallel^2}$ $\cdot\vec{u}_2$  \\\\
Por lo tanto $\{\vec{u}_1,\vec{u}_2,\vec{u}_3\}$ forman una base ortogonal de V. (¿Por qué seguro es base de V?)\\
Como queremos una base ortonormal dividimos a cada vector por su norma y nos queda: \\\\
$\vec{w}_1 = \dfrac{\vec{u}_1}{\parallel\vec{u}_1\parallel}$\indent 
$\vec{w}_2 = \dfrac{\vec{u}_2}{\parallel\vec{u}_2\parallel}$\indent 
$\vec{w}_3 = \dfrac{\vec{u}_3}{\parallel\vec{u}_3\parallel}$ \\\\
$\Rightarrow$ $\{\vec{w}_1,\vec{w}_2,\vec{w}_3\}$ es una base ortonormal de V.
\section{Nuevas Definiciones}
\subsection{Subespacios Ortogonales}
Dado S y T subespacios de V e.e. decimos que S es ortogonal a T (lo escribimos $S \perp T$)\\ si y sólo si $\vec{x} \perp \vec{y}$  $\forall$ $\vec{x} \in S$ , $\forall$ $\vec{y} \in T$
\subsection{Complemento Ortogonal de un subespacio}
Dado S subespacio de V e.e. llamaremos\\
$S^{\perp}$ = $\{\vec{x} \in V / \vec{x} \perp \vec{s}$ $\forall$ $\vec{s} \in V\}$ 
\subsection{Proposición:}
Dado S subespacio de V e.e. entonces\\
$S \oplus S^{\perp} $ = V \\
Debemos probar: \\
a) $S \oplus S^{\perp} \subset V$. Esto es evidente (¿Por qué?)\\
b) $V \subset S \oplus S^{\perp}$\\
Sea B = $\{\vec{s}_1,\vec{s}_2,\hdots,\vec{s}_k\}$ una base ortogonal de S. Extendemos B a una base ortogonal de V, y nos queda:\\
B' = $\{\vec{s}_1,\vec{s}_2,\hdots,\vec{s}_k,\vec{a}_1,\vec{a}_2,\hdots,\vec{a}_r\}$ base ortogonal de V. \\
Vamos a probar que $\{\vec{a}_1,\vec{a}_2,\hdots,\vec{a}_r\}$ es una base ortogonal de $S^{\perp}$\\
Que es un conjunto L.I. es obvio (¿Por qué?)\\
Veamos entonces que es un S.G. de $S^{\perp}$:\\
Sea $\vec{x} \in S^{\perp} \Rightarrow \vec{x} \in V \Rightarrow$
\begin{equation}
\vec{x} = \alpha_1\vec{s}_1 + \alpha_2\vec{s}_2+\hdots+\alpha_k\vec{s}_k+\beta_1\vec{a}_1+\beta_2\vec{a}_2+\hdots+\beta_r\vec{a}_r
\end{equation}
Sabemos que $(\vec{x}\cdot\vec{s}_i)=0$ $\forall i$ con $1 \leq i \leq k $ (¿Por qué?)\\
Entonces multiplicamos ambos miembros de (1) por $\vec{s}_1$ y nos queda: \\
$0=\alpha_1(\vec{s}_1\cdot\vec{s}_1) + \hdots + \alpha_k(\vec{s}_1\cdot\vec{s}_k) + \beta_1(\vec{s}_1\cdot\vec{a}_1) + \hdots + \beta_r(\vec{s}_1\cdot\vec{a}_r)$\\
$0=\alpha_1\cdot\parallel\vec{s}_1\parallel^2 + \hdots + \alpha_k\cdot(0) + \beta_1\cdot(0) + \hdots + \beta_r\cdot(0)$\\
$\Rightarrow$ nos queda $\alpha_1=0$\\
De la misma manera demostraremos que $\alpha_i = 0$ $\forall i$   con $1 \leq i \leq k$\\
Por lo tanto $\vec{x} = \beta_1\vec{a}_1+\beta_2\vec{a}_2+\hdots+\beta_r\vec{a}_r$ \\
$\Rightarrow$ $\{\vec{a}_1,\vec{a}_2,\hdots,\vec{a}_r\}$ es S.G. de $S^{\perp}$\\
Ahora sea $\vec{w} \in V$ $\Rightarrow$\\
$\vec{w} = (\theta_1\vec{s}_1 + \theta_2\vec{s}_2+\hdots+\theta\vec{s}_k)+(\pi_1\vec{a}_1+\pi_2\vec{a}_2+\hdots+\pi_r\vec{a}_r)$\\
$\vec{w}=\vec{c}+\vec{d}$ con $\vec{c} \in S$ y $\vec{d} \in S^{\perp}$\\
$\Rightarrow \vec{w} \in S \oplus S^{\perp}$ o sea $ \subset S \oplus S^{\perp}$
Nos falta probar que $S \cap S^{\perp} = \{\vec{0}\}$\\
Como S y $S^{\perp}$ son subespacios de V y la intersección de subespacios es un subespacio entonces $\vec{0} \in S \cap S^{\perp}$.\\
Sea $\vec{x} \in S \cap S^{\perp}$ $\Rightarrow$ $\vec{x} \in S$ y $\vec{x} \in S^{\perp}$\\
$\Rightarrow$ $(\vec{x} \cdot \vec{x}) = 0$
$\Rightarrow$ por la propiedad 4) de P.I., $\vec{x} = 0$\\
\textbf{Corolario:} De esta proposición se deduce que: \\
Dim V = dim $(S \oplus S^{\perp})$ = dim S + dim $S^{\perp}$
\section{Proyección ortogonal de un vector respecto a un subespacio}
Dado $\vec{x} \in V$ e.e. y S $\subseteq$ V, puedo escribir\\
$\vec{x}$ = $\vec{u} + \vec{v}$ con $\vec{u} \in S$, $\vec{v} \in S^{\perp}$ (¿Por qué?)\\
Llamamos $\vec{u}$ = $proy_S(\vec{x})$ y $\vec{v}$ = $proy_{S^{\perp}}(\vec{x})$
\subsection{Teorema de la mejor aproximación}
Si S es un subespacio de V e.e , y $\vec{x} \in V$ entonces $proy_S(\vec{x})$ es la mejor aproximación para $\vec{x}$ en S, o sea\\
$\parallel \vec{v} - \vec{x} \parallel$ $\geq$ $\parallel proy_S(\vec{x}) - \vec{x} \parallel$ $\forall$ $\vec{v} \in S$\\
\textbf{Hipótesis:} V e.e. S subespacio de V, $\vec{x} \in V$ (elemento cualquiera de V) \\
\textbf{Tésis:} $\parallel \vec{v} - \vec{x} \parallel$ $\geq$ $\parallel proy_S(\vec{x}) - \vec{x} \parallel$ $\forall$ $\vec{v} \in S$\\
\textbf{Demostración:} Vamos a trabajar con las normas al
cuadrado (¿Por qué podemos hacerlo?)\\
Sea $\vec{v} \in S$ $\Rightarrow$\\
$\parallel \vec{v} - \vec{x} \parallel^2$ = $\parallel \vec{v} - proy_S(\vec{x}) + proy_S(\vec{x}) - \vec{x} \parallel^2$ \\
$\parallel \vec{v} - \vec{x} \parallel^2$ = $\parallel \vec{v} - proy_S(\vec{x}) \parallel^2$ + $\parallel proy_S(\vec{x})- \vec{x} \parallel^2$ (¿Por qué?)\\
Y por lo tanto: \\
$\parallel \vec{v} - proy_S(\vec{x}) \parallel^2$ + $\parallel proy_S(\vec{x})- \vec{x} \parallel^2$ $\geq$ $\parallel proy_S(\vec{x})- \vec{x} \parallel^2$ (¿Por qué?)
\newpage
\subsection{Método de cuadrados mínimos para resolver sistemas lineales incompatibles}
Hemos probado en un ejercicio de la práctica que un sistema lineal de la forma $A\cdot\vec{X}=\vec{B}$ es compatible $\leftrightarrow$ $\vec{B} \in col(A)$ siendo $col(A)$ = $<\vec{C}_1, \vec{C}_2, \ldots, \vec{C}_n>$ donde $\vec{C}_i$ son las columnas de la matriz A. \\
Por lo tanto si $A\cdot\vec{X}=\vec{B}$ es un sistema incompatible, es seguro que $\vec{B} \notin col(A)$. Luego, la única manera de convertirlo en compatible consiste en buscar un vector perteneciente a $col(A)$, cometiendo el menor error posible. \\
¿Cuál será este vector? Según lo demostrado anteriormente, este vector debe ser $proy_{col(A)}(\vec{B})$ (¿Por qué?)\\
Osea cuando un sistema lineal $A\cdot\vec{X}=\vec{B}$ es incompatible, la mejor solución aproximada que podemos encontrar es buscando la solución de: \\
$A\cdot\vec{X}=proy_{col(A)}(\vec{B})$
\subsubsection{Fórmula para resolver un sistema lineal incompatible}
Para llegar a la fórmula debemos probar:\\
\textbf{a)} $(col(A))^{\perp}$ = $\{\vec{X} \in \rm I\!R^n$ / $A^T\cdot\vec{X}=0\}$\\
Dado $col(A)$ = $<\vec{C}_1, \vec{C}_2, \ldots, \vec{C}_n>$ $\rightarrow$\\
$\vec{X} \in (col(A))^{\perp}$ $\leftrightarrow$ $(\vec{X}\cdot\vec{C}_i)=0$ $\forall$ $1 \leq i \leq n$, pero:\\\\
$A = \begin{pmatrix}
\uparrow & \uparrow & \ldots & \uparrow\\ 
\vec{C}_1 & \vec{C}_2 & \ldots & \vec{C}_n\\
\downarrow & \downarrow & \ldots & \downarrow 
\end{pmatrix}$
\indent $\leftrightarrow$ \indent
$A^T = \begin{pmatrix}
\leftarrow & \vec{C}_1 & \rightarrow\\ 
\leftarrow & \vec{C}_2 & \rightarrow\\
\vdots & \vdots & \vdots \\
\leftarrow & \vec{C}_n & \rightarrow 
\end{pmatrix}$\\\\
\textbf{b)} Demostrar que si $\vec{X}_0$ es solución por el
método de cuadrados mínimos del sistema $A\cdot\vec{X}=\vec{B}$, entonces $(A\cdot\vec{X}_0-\vec{B})$ $\in (col(A))^{\perp}$\\
Si $\vec{X}_0$ es solución del sistema incompatbile $A\cdot\vec{X}=\vec{B}$ $\rightarrow$\\
\begin{equation}
A\cdot\vec{X}_0=proy_{col(A)}(\vec{B})
\end{equation}
Restando en (3) $\vec{B}$ a ambos miembros nos queda:\\
$A\cdot\vec{X}_0 - \vec{B}=proy_{col(A)}(\vec{B}) - \vec{B}$ = $proy_{col(A)^{\perp}}(\vec{B})$ (¿Por qué?)\\
Por lo tanto, $A\cdot\vec{X}_0 - \vec{B}$ $\in$ $(col(A))^{\perp}$
\newpage
\noindent \textbf{c)} Demostrar que $A^T (A \cdot \vec{X}_0) = A^T \cdot \vec{B}$\\
Si $A \cdot \vec{X}_0 - \vec{B}$ $\in$ $(col(A))^{\perp}$\\
$\rightarrow$ $A^T (A\cdot\vec{X}_0 - \vec{B}) = 0$\\
$\rightarrow$ $A^T (A\cdot\vec{X}_0) = A^T \cdot \vec{B}$ (¿Por qué?)
\end{document}