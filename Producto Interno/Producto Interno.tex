\documentclass{article}
\usepackage[utf8]{inputenc}
\usepackage{amsmath}
\usepackage{color}
% -------------  Margenes ---------------
\addtolength{\oddsidemargin}{-.875in}
	\addtolength{\evensidemargin}{-.875in}
	\addtolength{\textwidth}{1.75in}

	\addtolength{\topmargin}{-.875in}
	\addtolength{\textheight}{1.75in}
% ------------- /Margenes ---------------
% ------------- Espaciado entre lineas ---------------
\renewcommand{\baselinestretch}{1.5} 
% !TeX spellcheck = es_ES
\title{Producto Interno}
\author{Álgebra II}
\date{\vspace{-5ex}}

\begin{document}
\maketitle{}
% ------------- Comienza el documento ---------------
\section{Definiciones}
\subsection{Producto Interno}
\textbf{Definición:} Un producto interno o escalar sobre un espacio vectorial real V  es una función \\
p: $VxV$ $\rightarrow \rm I\!R $ que cumple con las siguientes propiedades:
\begin{enumerate}
\item $(\vec{x}\cdot\vec{y})=(\vec{y}\cdot\vec{x})$ 
\item $((\vec{x}+\vec{y})\cdot\vec{z})=(\vec{x}\cdot\vec{z})+(\vec{y}\cdot\vec{z})$
\item $(\lambda\cdot\vec{x}\cdot\vec{y}) = \lambda\cdot(\vec{x}\cdot\vec{y})$
\item I) $ $ $(\vec{x}\cdot\vec{x} \geq 0)$ \\ II) $(\vec{x}\cdot\vec{x}) = 0 \leftrightarrow \vec{x}=0 $
\end{enumerate}
\subsection{Espacio Euclideo}
Si V es un e.v. en el que se ha definido un producto interno recibe el nombre de espacio euclídeo. 
\subsection{Definiciones en un espacio euclideo}
\begin{itemize}
\item Norma de un vector: $\parallel\vec{x}\parallel^2 = (\vec{x}\cdot\vec{x})$\\
a) Con cualquier producto interno, el único vector de norma cero es el $\vec{0}$ (¿Por qué?)\\
b) $\parallel\vec{x}\parallel$ = $\parallel-\vec{x}\parallel$ (¿Por qué?)\\
c) $\parallel\alpha\cdot\vec{x}\parallel$ = $\mid\alpha\mid\cdot\parallel\vec{x}\parallel$ (¿Por qué?)
\item Distancia entre dos vectores: \\$dist(\vec{x},\vec{y})$ = $\parallel\vec{x}-\vec{y}\parallel$
\item Ángulo entre dos vectores: \\$\cos \phi = \dfrac{\vec{x}\cdot\vec{y}}{\parallel\vec{x}\parallel\cdot\parallel\vec{y}\parallel}$
\item Ortogonalidad entre vectores: \\
Dos vectores $\vec{x}$ e $\vec{y}$ son ortogonales $\leftrightarrow$ $\vec{x}\cdot\vec{y}$ = $0$
\newpage
\item Conjunto Ortogonal de Vectores: \\
Un conjunto $\{\vec{e}_1, \vec{e}_2, \hdots, \vec{e}_n\}$ es un conjunto ortogonal $\leftrightarrow$ $\vec{e}_i \neq \vec{0}$ $\forall i$ y $(\vec{e}_i\cdot\vec{e}_j)$ = $0$ $\forall i \neq j$
\item Conjunto Ortonormal de Vectores: \\
Un conjunto $\{\vec{e}_1, \vec{e}_2, \hdots, \vec{e}_n\}$ es ortonormal $\leftrightarrow$ es ortogonal y la norma de sus vectores es 1.
\end{itemize}
% ----------------------- Proposiciones ----------------------
\section{Proposiciones}
\subsection{Teorema de Pitágoras}
Cualquiera sean vectores $\vec{x}$ e $\vec{y}$ vectores pertenecientes a $V e.e.$ tal que $\vec{x}\perp\vec{y}$ se cumple: \\
\centerline {$\parallel\vec{x}+\vec{y}\parallel^2$ = $\parallel\vec{x}\parallel^2$ + $\parallel\vec{y}\parallel^2$}\\
\textbf{Hipótesis:} V e.e. $\vec{x}, \vec{y} \in V$ $\vec{x}\perp\vec{y}$\\
\textbf{Tésis:} $\parallel\vec{x}+\vec{y}\parallel^2$ = $\parallel\vec{x}\parallel^2$ + $\parallel\vec{y}\parallel^2$ \\
\textbf {Demostración:}\\
$\parallel\vec{x}+\vec{y}\parallel^2$ = $(\vec{x} + \vec{y})\cdot(\vec{x} + \vec{y}) \rightarrow$ Expreso la norma como producto \\ 
 = $(\vec{x}\cdot\vec{x}) + (\vec{x}\cdot\vec{y}) + (\vec{y}\cdot\vec{x})+ (\vec{y}\cdot\vec{y}) \rightarrow$ Distribuyo por propiedad 2 \\
 = $\parallel\vec{x}\parallel^2$+  0  +  0  +$\parallel\vec{y}\parallel^2$ (¿Por qué?) \\
O sea hemos llegado a la tesis
\subsection{Desigualdad de Schwarz}
Cualquiera sea $\vec{x}$ e $\vec{y} \in $ V e.e. se cumple: \\
$\mid(\vec{x}\cdot\vec{y})\mid$ $\leq$ $\parallel\vec{x}\parallel \cdot \parallel\vec{y}\parallel $ \\
Esta demostración es necesaria para que esté bien definido el ángulo entre dos vectores (¿Por qué?)\\
\textbf{Hipótesis:} V e.e. $\vec{x}$ e $\vec{y} \in V$ \\
\textbf{Tésis:} $\mid(\vec{x}\cdot\vec{y})\mid$ $\leq$ $\parallel\vec{x}\parallel \cdot \parallel\vec{y}\parallel $ \\
\textbf {Demostración:} \\
Para realizar esta demostración, definimos: $\vec{z} = \vec{x} + \alpha\cdot\vec{y}$ \\
Entonces podemos afirmar $((\vec{x} + \alpha\cdot\vec{y})\cdot(\vec{x} + \alpha\cdot\vec{y})) \geq 0$\\
Aplicando propiedades de producto interno (¿Cuales?) nos queda: \\
$\parallel\vec{x}\parallel^2 +$ 2 $\alpha\cdot(\vec{x}\cdot\vec{y}) + \alpha^2 \cdot \parallel\vec{y}\parallel^2$ $\geq 0$\\
Esta expresión representa geométricamente una parábola una parábola de variable $\alpha$, la cual es siempre $\geq$ 0. Entonces tiene 1 o ninguna o 
\end{document}