\documentclass{article}
\usepackage[utf8]{inputenc}
\usepackage{amsmath}
\usepackage{color}
% -------------  Margenes ---------------
\addtolength{\oddsidemargin}{-.875in}
	\addtolength{\evensidemargin}{-.875in}
	\addtolength{\textwidth}{1.75in}

	\addtolength{\topmargin}{-.875in}
	\addtolength{\textheight}{1.75in}
% ------------- /Margenes ---------------
% ------------- Espaciado entre lineas ---------------
\renewcommand{\baselinestretch}{1.5} 
% !TeX spellcheck = es_ES
\title{Espacios Vectoriales}
\author{Álgebra II}
\date{\vspace{-5ex}}

\begin{document}
\maketitle{}
% ------------- Comienza el documento ---------------
\section{Definición}
Sea V un conjunto cuyos elementos se llamarán vectores en
el cual se definen dos operaciones :

\begin{itemize}
\item Suma de vectores
\item Producto de un vector por un escalar k $\in {\rm I\!R}$\
\end{itemize}
Estas operaciones cumplen las siguientes propiedades:\\\\
\underline {Propiedades de la Suma:}
\begin{enumerate}
\item {\bfseries Cerrada:} Si $\vec{u}, \vec{v} \in V \Rightarrow \vec{u}+\vec{v} \in V$
\item {\bfseries Conmutativa:} $\vec{u} + \vec{v} = \vec{v} + \vec{u}$ para todo $\vec{u}, \vec{v} \in V $ 
\item {\bfseries Asociativa:} $(\vec{u}+\vec{v})+\vec{w} = \vec{u}+(\vec{v}+\vec{w})$
\item {\bfseries Elemento Neutro:} $\exists$ $\vec{0} \in V$ / $\vec{0} + \vec{x}$ = $\vec{x}$ $\forall$ $\vec{x}  \in V$
\item {\bfseries Vector Inverso:} $\forall$ $\vec{x} \in$ V, existe un vector inverso -$\vec{x}$ / $\vec{x}+(-\vec{x})=\vec{0}$
\end{enumerate}
\underline{Propiedades del producto:}
\begin{enumerate}
\item {\bfseries Cerrada:} Si $\vec{x} \in V \Rightarrow k\cdot\vec{x} \in V$
\item {\bfseries Neutro:} $1\cdot\vec{x} = \vec{x}$ $\forall$ $\vec{x} \in V $ 
\item {\bfseries Asociativa:} $k_{1}\cdot(k_{2} \cdot \vec{x} ) = (k_{1} \cdot k_{2})\cdot\vec{x}$
\item {\bfseries Distributiva:} $k  \cdot (\vec{x} + \vec{y}) = k \cdot \vec{x} + k \cdot \vec{y} $
\item {\bfseries Distributiva:} $(k_{1} + k_{2})\cdot \vec{x} = k_{1}\cdot\vec{x} + k_{2}\cdot\vec{x}$
\end{enumerate}
\newpage
% ------------- Subespacios ---------------
\section{Subespacios} 
{\bfseries Definición:} Un subconjunto S no vacío de V e.v.es un subespacio de V si la suma y el producto definidas en  V estructuran también a S como un espacio vectorial \\
\underline {Propiedades necesarias para que S $\subseteq$ V sea subespacio}
\begin{enumerate}
\item Si $\vec{x} \in S$ e $\vec{y} \in S \Rightarrow \vec{x} + \vec{y} \in S$
\item Si $\vec{x} \in S \Rightarrow k \cdot \vec{x} \in S$
\item $\vec{o} \in S$
\end{enumerate}
\subsection{Definiciones}
\begin{enumerate}
\item {\bfseries Combinación Lineal:} $\vec{x}$ es una combinación lineal de los vectores $\vec{x}_{1}, \vec{x}_2,... \vec{x}_k $ si existen escalares $k_{1}, k_{2},... k_{k} $ tal que $\vec{x} = k_{1}\vec{x}_{1} + k_{2}\vec{x}_{2} +... k_{k}\vec{x}_{k}$
\item {\bfseries Sistema de Generadores:} Un conjunto de vectores M = \{$\vec{x}_{1}, \vec{x}_2,... \vec{x}_k $\} es un sistema de generadores de V $\Leftrightarrow \forall $ $ \vec{v} \in V, \vec{v} = \alpha_{1}\vec{x}_{1} + \alpha_{2}\vec{x}_{2} +... \alpha_{k}\vec{x}_{k}$
\item {\bfseries Conjunto de vectores linealmente independiente y linealmente dependiente:}\\
Sea M = \{ $\vec{v}_{1}, \vec{v}_2,... \vec{v}_k $ \} tal que $\vec{v}_{i} \in$ V $\forall $ i, y sea $\alpha_{1}\vec{x}_{1} + \alpha_{2}\vec{x}_{2} +... \alpha_{k}\vec{x}_{k} = \vec{0}$ (o sea una combinación lineal de ellos igualada a $\vec{0}$)\\
Nos queda entonces un sistema lineal homogéneo que puede tener:\\\\
a) \underline{Solución Única:} La unica solución es la trivial, por lo que M es un conjunto {\bfseries Linealmente Independiente (L.I.)}\\
b) \underline{Infinitas Soluciones:} M es un conjunto {\bfseries Linealmente Dependiente (L.D.)}
\end{enumerate}
\newpage
% ------------- Base ---------------
\section{Base de un Espacio Vectorial}
{\bfseries Definición:} Dado V e.v. y B = \{$\vec{v}_{1}, \vec{v}_2,... \vec{v}_n $\} tal que $\forall$ $\vec{v}_{i} \in$ V decimos que B es una base de V si y sólo si se cumple:\\\\
a) B es un conjunto L.I.\\
b) B es un sistema de generadores de V \\\\
Ejemplos de bases canónicas: 
\begin{itemize}
\item {\bfseries En R$^{n}$:} (1,0,0,...,0), (0,1,0,...,0),... (0,0,0...,1) 
\item {\bfseries En R$^{2x2}$:} $\begin{pmatrix} 1&0\\ 0&0 \end{pmatrix}$ , $\begin{pmatrix} 0&1\\ 0&0 \end{pmatrix}$ , $\begin{pmatrix} 0&0\\ 1&0 \end{pmatrix}$ , $\begin{pmatrix} 0&0\\ 0&1 \end{pmatrix}$ 
\item {\bfseries En P$_{n}$:} \{1, x, x$^2$,... x$^n$ \} 
\end{itemize}
% ------------- Coordenadas ---------------
\section{Coordenadas de un vector respecto de una base}
Dado B = \{$\vec{e}_{1}, \vec{e}_2,... \vec{e}_n $\} una base de V (e.v.) y dado $\vec{x} \in V$\\ 
$\Rightarrow$ Se denomina a los escalares $\lambda_{1}, \lambda_{2},... \lambda_{n}$ tal que $\vec{x} = \lambda_{1}\vec{e}_{1} + \lambda_{2}\vec{e}_{2} +... \lambda_{k}\vec{e}_{k}$ como {\bfseries coordenadas de} $\vec{x} $ respecto de la base B y se escribe $\vec{x}]_B$ \\
Por lo tanto: $\vec{x}]_B$ = ($\lambda_{1}, \lambda_{2},... \lambda_{n}$)
\newpage
% ------------- Proposiciones ---------------
\section{Proposiciones}
% ------------- Coordenadas Unicas ---------------
\subsection{Las coordenadas de un vector respecto de una base 
son únicas}
{\bfseries \textcolor{blue}{Hipótesis:}} V e.v. \{$\vec{e}_{1}, \vec{e}_2,... \vec{e}_n $\} base de V, $\vec{x} \in V$ \\
{\bfseries \textcolor{red}{Tesis:}} Existe una única n-upla ($\alpha_{1}, \alpha_{2},... \alpha_{n}$) tal que $\vec{x} = \alpha_{1}\vec{e}_{1} + \alpha_{2}\vec{e}_{2} +... \alpha_{n}\vec{e}_{n}$\\
{\bfseries Demostración:} Supongo que existen 2 n-uplas que satisfacen la definición de coordenadas respecto de una base, es decir:\\
$\vec{x} = \alpha_{1}\vec{e}_{1} + \alpha_{2}\vec{e}_{2} +... \alpha_{n}\vec{e}_{n} = \beta_{1}\vec{e}_{1} + \beta_{2}\vec{e}_{2} +... \beta_{n}\vec{e}_{n}$\\
Pasando al segundo miembro y sacando factor común $\vec{e}_i$ nos queda: \\
$(\alpha_{1} - \beta_{1})\cdot\vec{e}_{1} + (\alpha_{2} - \beta_{2})\cdot\vec{e}_{2} +... (\alpha_{n} -\beta_{n})\cdot\vec{e}_{n} = \vec{0}$\\
Como \{$\vec{e}_{1}, \vec{e}_2,... \vec{e}_n $\} es una base de V, es decir se trata de vectores L.I. Esto significa que si tengo una combinación lineal de ellos igualada a $\vec{0}$, sus escalares son = 0. \\
Por lo tanto, $\alpha_{i} = \beta_{i}$ $\forall$ i.
% ------------- n+1 es LD --------------- 
\subsection{Si un espacio vectorial tiene una  base de n 
elementos , entonces cualquier conjunto de n+1 
elementos es un conjunto l.d.}
{\bfseries \textcolor{blue}{Hipótesis:}} V e.v. B = \{$\vec{e}_{1}, \vec{e}_2,... \vec{e}_n $\} base de V, \{$\vec{x}_{1}, \vec{x}_2,... \vec{x}_n, \vec{x}_{n+1} $\} conjunto de n+1 elementos\\
{\bfseries \textcolor{red}{Tesis:}} \{$\vec{x}_{1}, \vec{x}_2,... \vec{x}_n, \vec{x}_{n+1} $\} es L.D.\\
{\bfseries Demostración:} Para demostrar si un conjunto es l.i.  debo probar 
que cualquier combinación lineal de sus vectores igualada a $\vec{0}$ tiene una única solución (La solución trivial). Para probar que es L.D., una combinación lineal de dichos vectores igualada a $\vec{0}$ tendrá infinitas soluciones. \\
\end{document}
