\documentclass{article}
\usepackage[utf8]{inputenc}
% !TeX spellcheck = es_ES
\title{Espacios Vectoriales}
\author{Álgebra II}
\date{\vspace{-5ex}}

\begin{document}
\maketitle{}
% ------------- Comienza el documento ---------------
\section{Definición}
Sea V un conjunto cuyos elementos se llamarán vectores en
el cual se definen dos operaciones :

\begin{itemize}
\item Suma de vectores
\item Producto de un vector por un escalar k $\in {\rm I\!R}$\
\end{itemize}
Estas operaciones cumplen las siguientes propiedades:
\begin{enumerate}
\item {\bfseries Cerrada:} Si $\vec{u}, \vec{v} \in V \Rightarrow \vec{u}+\vec{v} \in V$
\item {\bfseries Conmutativa:} $\vec{u} + \vec{v} = \vec{v} + \vec{u}$ para todo $\vec{u}, \vec{v} \in V $ 
\item {\bfseries Asociativa:} $(\vec{u}+\vec{v})+\vec{w} = \vec{u}+(\vec{v}+\vec{w})$
\item {\bfseries Elemento Neutro:} $\exists$ $\vec{0} \in V$ / $\vec{0} + \vec{x}$ = $\vec{x}$ $\forall$ $\vec{x}  \in V$
\item {\bfseries Vector Inverso:} $\forall$ $\vec{x} \in$ V, existe un vector inverso -$\vec{x}$ / $\vec{x}+(-\vec{x})=\vec{0}$
\end{enumerate}
\end{document}
